
\section{Getting started}\label{sec:start}\index{PISM!getting started}

\subsection{On installing PISM}

To install PISM see the \emph{Getting PISM} tab at \href{http://www.pism-docs.org}{\texttt{www.pism-docs.org}}.  Or get the PISM Installation Manual (PDF) at \url{http://www.pism-docs.org/wiki/lib/exe/fetch.php?media=installation.pdf}.  Once PISM is installed, the executable \texttt{pismr} should be available on your system's ``path''; confirm this with ``\texttt{which pismr}''.  The instructions below assume you are using a \texttt{bash} shell (or one that accepts \texttt{bash} syntax).  They also assume you have the PISM source code in the directory ``\texttt{pism/}''.

\subsection{A Greenland ice sheet example}

We get started with an extended example showing how to generate initial states for prognostic model experiments on the Greenland ice sheet.  Ice sheet and glacier model studies often involve modeling present and past states using actions like the ones demonstrated here.  Our particular choices made here are motivated by the evaluation of initialization methods in \cite{AschwandenAdalgeirsdottirKhroulev}.

We use data assembled by the \href{http://websrv.cs.umt.edu/isis/index.php/SeaRISE_Assessment}{Sea-level Response to Ice Sheet Evolution (SeaRISE)} group \cite{Bindschadler2013SeaRISE}.  SeaRISE is a community-organized assessment process providing an upper bound on ice sheet contributions to sea level in the next 100--200 years, especially for the IPCC AR5 report in 2013.

This example is a hands-on first look at PISM.  It is not an in-depth tutorial, and some details of what is happening are only explained later in this Manual.  The remainder of this Manual discusses PISM options, nontrivial modeling choices, and how to preprocess input data.

The basic runs here, mostly on coarse $20$ and $10\,\textrm{km}$ grids, can be done on a typical workstation or laptop.  PISM is, however, designed to make high resolution (e.g.~$5\,\textrm{km} \to 1\,\textrm{km}$ grids for whole-Greenland ice sheet modeling) possible by exploiting large-scale parallel processing.  See \cite{AschwandenAdalgeirsdottirKhroulev,Golledgeetal2012,Golledgeetal2013}, among other published PISM examples.


\subsection{Input data}

The NetCDF data used to initialize SeaRISE runs is freely-available online: 
\medskip

\centerline{\protect{\textbf{\url{http://websrv.cs.umt.edu/isis/index.php/Present_Day_Greenland}}}}
\medskip

\noindent To download the specific file we want, namely \texttt{Greenland_5km_v1.1.nc}, and preprocess it for PISM, do:
\begin{verbatim}
$ cd pism/examples/std-greenland
$ ./preprocess.sh
\end{verbatim}
\noindent The script \texttt{preprocess.sh} requires \texttt{wget} and also the NetCDF Operators (``NCO''; \url{http://nco.sourceforge.net/}).  It downloads the version 1.1 of the SeaRISE ``master'' present-day data set, which contains ice thickness and bedrock topography from BEDMAP \cite{BamberLayberryGogenini}, and modeled precipitation and surface mass balance rates from RACMO \cite{Ettemaetal2009}, among other fields.

In particular, it creates three new NetCDF files which can be read by PISM.  The spatially-varying fields, with adjusted metadata, go in \texttt{pism_Greenland_5km_v1.1.nc}.  The other two new files contain famous time-dependent paleo-climate records from ice and seabed cores: \texttt{pism_dT.nc} has the GRIP temperature record \cite{JohnsenetalGRIP} and \texttt{pism_dSL.nc} has the SPECMAP sea level record \cite{Imbrieetal1984}.

Any of these NetCDF files can be viewed with \texttt{ncview} or other NetCDF visualization tools; see Table \ref{tab:NetCDFview} below.  An application of IDV to the master data set produced Figure \ref{fig:sr-input}, for example.  Use \texttt{ncdump -h} to see the metadata and history of the files.

\begin{figure}[ht]
\centering
\mbox{\includegraphics[width=2.0in,keepaspectratio=true]{sr-greenland-thk}
  \qquad
  \includegraphics[width=2.0in,keepaspectratio=true]{sr-greenland-topg}
  \qquad
  \includegraphics[width=2.0in,keepaspectratio=true]{sr-greenland-prcp}}
\caption{The input file contains present-day ice thickness (left; m), bedrock elevation (center; m), and present-day precipitation (right; m $\text{a}^{-1}$ ice equivalent) for SeaRISE-Greenland.  These are fields \texttt{thk}, \texttt{topg}, and \texttt{precipitation}, respectively, in \texttt{pism_Greenland_5km_v1.1.nc}.}
\label{fig:sr-input}
\end{figure}


\subsection{First run}   \label{subsect:runscript}  Like many Unix programs, PISM allows a lot of command-line options.  In fact, because the variety of allowed ice sheet, shelf, and glacier configurations, and included sub-models, is so large, the list of possible command-line options covers sections \ref{sec:boot} through \ref{sec:practical-usage} of this manual.  In practice one often builds scripts to run PISM with the correct options, which is what we show here.  The script we use is ``\texttt{spinup.sh}'' in the \texttt{examples/std-greenland/} subdirectory of \texttt{pism/}.

Note that initializing ice sheets, generically called ``spin-up'', can be done by computing approximate steady states with constant boundary data, or, in some cases, by integrating paleo-climatic and long-time-scale information, also applied at the ice sheet boundary, to build a model for the present state of the ice sheet.  Both of these possibilities are illustrated in the \texttt{spinup.sh} script.  The spin-up stage of using an ice sheet model may actually require more processor-hours than follow-on ``experiment'' or ``forecast'' stages.

To see what can be done with the script, read the usage message it produces:
\begin{verbatim}
$ ./spinup.sh
\end{verbatim}

The simplest spin-up approach is to use a ``constant-climate'' model.  We take this approach first.  To see a more detailed view of the PISM command for the first run, do:
\begin{verbatim}
$ PISM_DO=echo ./spinup.sh 4 const 10000 20 sia g20km_10ka.nc
\end{verbatim}
Setting the environment variable \texttt{PISM_DO} in this way tells \texttt{spinup.sh} just to print out the commands it is about to run, not do them.  The ``proposed'' run looks like this:
\label{firstcommand}
\small
\begin{verbatim}
mpiexec -n 4 pismr -boot_file pism_Greenland_5km_v1.1.nc -Mx 76 -My 141 \
  -Mz 101 -Mbz 11 -z_spacing equal -Lz 4000 -Lbz 2000 -skip -skip_max 10 \
  -ys -10000 -ye 0 -surface given -surface_given_file pism_Greenland_5km_v1.1.nc \
  -ocean_kill pism_Greenland_5km_v1.1.nc -sia_e 3.0 \
  -ts_file ts_g20km_10ka.nc -ts_times -10000:yearly:0 \
  -extra_file ex_g20km_10ka.nc -extra_times -10000:100:0 \
  -extra_vars diffusivity,temppabase,tempicethk_basal,bmelt,tillwat,csurf,mask,thk,topg,usurf \
  -o g20km_10ka.nc
\end{verbatim}
\normalsize
Let's briefly deconstruct this run.

At the front is ``\texttt{mpiexec -n 4 pismr}''.  This means that the PISM executable \texttt{pismr} is run in parallel on four processes parallel standard (e.g.~cores) under the Message Passing Interface (``MPI''; \url{http://www.mcs.anl.gov/mpi/}).  Though we are assuming you have a workstation or laptop with at least 4 cores, this example will work with 1 to about 50 processors, with reasonably good scaling in speed.  Scaling can be good with more processors if we run at higher spatial resolution \cite{BBssasliding,DickensMorey2013}.  The executable name ``\texttt{pismr}'' stands for the standard ``run'' mode of PISM (in contrast to specialized modes described later in sections \ref{sec:verif} and \ref{sec:simp}).

Next, the proposed run uses option \texttt{-boot_file} to start the run by ``bootstrapping.'' This term describes the creation, by heuristics and highly-simplified models, of the mathematical initial conditions required for a deterministic, time-dependent ice dynamics model.  Then the options describe a $76\times 141$ point grid in the horizontal, which gives 20 km grid spacing in both directions.  Then there are choices about the vertical extent and resolution of the computational grid; more on those later.  After that we see a description of the time-axis, with a start and end time given: ``\texttt{-ys -10000 -ye 0}''.

Then we get the instructions that tell PISM to read the upper surface boundary conditions (i.e.~climate) from a file: ``\texttt{-surface given -surface_given_file pism_Greenland_5km_v1.1.nc}''.  For more on these choices, see subsection \ref{sec:climate-inputs}, and also the PISM Climate Forcing Manual.

Then there are a couple of options related to ice dynamics.  First is a minimal calving model which removes ice at the calving front location given by a thickness field in the input file (``\texttt{-ocean_kill}''); see subsection \ref{sec:calving} for this and other calving options).  Then there is a setting for enhanced ice softness (``\texttt{-sia_e 3.0}'').  See subsection \ref{sec:rheology} for more on this enhancement parameter, which we also return to later in the current section in a parameter study.

Then there are longish options describing the fields we want as output, including scalar time series (``\texttt{-ts_file ts_g20km_10ka.nc -ts_times -10000:yearly:0}''; see section \ref{sec:practical-usage}) and space-dependent fields (``\texttt{-extra_file ...}''; again see section \ref{sec:practical-usage}), and finally the named output file (``\texttt{-o g20km_10ka.nc}'').

Note that the modeling choices here are reasonable, but they are not the only way to do it! The user is encouraged to experiment; that is the point of a model.

Now let's actually get the run going:
\begin{verbatim}
$ ./spinup.sh 4 const 10000 20 sia g20km_10ka.nc &> out.g20km_10ka &
\end{verbatim}
\noindent The terminating ``\verb|&|'' asks unix to run the command in the background, so we can keep working in the current shell.  Because we have re-directed the text output (``\verb|&> out.g20km_10ka|''), PISM will show what it is doing in the text file \texttt{out.g20km_10ka}.  Using \texttt{less} is a good way to watch such a growing text-output file.


\subsection{Watching the first run}  \label{subsect:watchrun}  As soon as the run starts it creates time-dependent NetCDF files \texttt{ts_g20km_10ka.nc} and \texttt{ex_g20km_10ka.nc}.  In fact the latter file, which has spatially-dependent fields at each time, is created after the first 100 model years, a few wall clock seconds in this case.  The command \texttt{-extra_file ex_g20km_10ka.nc -extra_times -10000:100:0} adds a spatially-dependent ``frame'' at model times -9900, -9800, \dots, 0.

To look at the spatial-fields output graphically, do:
\begin{verbatim}
$ ncview ex_g20km_10ka.nc
\end{verbatim}
We see that \texttt{ex_g20km_10ka.nc} contains growing ``movies'' of the fields chosen by the \texttt{-extra_vars} option.  A frame of the ice thickness field \texttt{thk} is shown in Figure \ref{fig:growing} (left).

The time-series file \texttt{ts_g20km_10ka.nc} is also growing.  It contains spatially-averaged ``scalar'' diagnostics like the total ice volume or the ice-sheet-wide maximum velocity (variable \texttt{ivol} and \texttt{max_hor_vel}, respectively).  It can be viewed
\begin{verbatim}
$ ncview ts_g20km_10ka.nc
\end{verbatim}
The growing time series for \texttt{ivol} is shown in Figure \ref{fig:growing} (right).  Recall that our intention was to generate a minimal model of the Greenland ice sheet in approximate steady-state with a steady (constant-in-time) climate.  The measurable steadiness of the \texttt{ivol} time series is a possible standard for steady state \cite[for example]{EISMINT00}.  The reader will probably agree at this point that PISM can produce quite a bit of information about a run, as the run is happening and in a user-controllable manner.

\begin{figure}[ht]
\centering
\mbox{\includegraphics[height=5.0in,keepaspectratio=true]{ex-growing-thk-g20km}
  \qquad \includegraphics[height=5.0in,keepaspectratio=true]{ts-growing-ivol-g20km}}
\caption{Two views produced by \texttt{ncview} during a PISM model run.  Left: \texttt{thk}, the ice sheet thickness, a space-dependent field, from file \texttt{ex_g20km_10ka.nc}.  Right: \texttt{ivol}, the total ice sheet volume time-series, from file \texttt{ts_g20km_10ka.nc}.}
\label{fig:growing}
\end{figure}

At the end of the run the output file \texttt{g20km_10ka.nc} is generated.  Figure \ref{fig:firstoutput} shows some fields from this file.  In the next subsections we consider their ``quality'' as model results.

To see a report on computational  performance, we do:
\begin{verbatim}
$ ncdump -h g20km_10ka.nc |grep history
    :history = "user@machine 2013-11-23 15:57:22 AKST: PISM done.  Performance stats:
0.3435 wall clock hours, 1.3738 proc.-hours, 7274.0065 model years per proc.-hour,
PETSc MFlops = 0.03.\n",
\end{verbatim}
The whole first run took about 21 minutes (0.3435 wall clock hours) on a 2012-era 4-core laptop.

\begin{figure}[ht]
\centering
\mbox{\includegraphics[height=2.75in,keepaspectratio=true]{g20km-10ka-usurf} \includegraphics[height=2.75in,keepaspectratio=true]{g20km-10ka-csurf} \includegraphics[height=2.75in,keepaspectratio=true]{g20km-10ka-mask}}
\caption{Fields from output file \texttt{g20km_10ka.nc}.  Left: \texttt{usurf}, the ice sheet surface elevation in meters.  Middle: \texttt{csurf}, the surface speed in meters/year ($=$ m/a), including the 100 m/a contour (solid black).  Right: \texttt{mask}, with 0 = ice-free land, 2 = ice, 4 = ice-free ocean.}
\label{fig:firstoutput}
\end{figure}


\subsection{Second run: a better ice-dynamics model}  \label{subsect:ssarun}

It is widely-understood that ice sheets slide on their bases, especially when liquid water is present at the base (see \cite{Joughinetal2001,MacAyeal}, among others).  An important aspect of modeling such sliding is the inclusion of membrane or ``longitudinal'' stresses into the stress balance \cite{BBssasliding}.  The basic stress balance in PISM which involves membrane stresses is the Shallow Shelf Approximation (SSA) \cite{WeisGreveHutter}.  The stress balance used in the previous section was, by contrast, the (thermomechanically-coupled) non-sliding, non-membrane-stress Shallow Ice Approximation (SIA) \cite{BBL,EISMINT00}.  The preferred ice dynamics model within PISM, that allows both sliding balanced by membrane stresses and shear flow as described by the SIA, is the SIA+SSA ``hybrid'' model \cite{BBssasliding,Winkelmannetal2011}.  For more on stress balance theories see section \ref{sec:dynamics} of this Manual.

A practical issue with models of sliding is that a distinctly-uncertain parameter space must be introduced.  This especially involves parameters controlling the amount and pressure of subglacial water (see \cite{AschwandenAdalgeirsdottirKhroulev,Clarke05,Tulaczyketal2000,vanPeltOerlemans2012} among other references).  In this regard, PISM uses the concept of a saturated and pressurized subglacial till with a modeled distribution of yield stress  \cite{BBssasliding,SchoofStream}.  The yield stress arises from the PISM model of the production of subglacial water, which is itself computed through the conservation of energy model \cite{AschwandenBuelerKhroulevBlatter}.  We use such models in the rest of this Getting Started section.

While the \texttt{spinup.sh} script has some default sliding-related parameters, for demonstration purposes we change one such parameter.  We replace the default power $q=0.25$ in the sliding law (the equation which relates both the subglacial sliding velocity and the till yield stress to the basal shear stress which appears in the SSA stress balance) by a less ``plastic'' and more ``linear'' choice $q=0.5$.  See subsection \ref{subsect:basestrength} for more on sliding laws.

To see the run we propose, do
\begin{verbatim}
$ PISM_DO=echo PARAM_PPQ=0.5 ./spinup.sh 4 const 10000 20 hybrid g20km_10ka_hy.nc
\end{verbatim}
Now remove ``\texttt{PISM_DO=echo}'' and redirect the text output into a file to start the run:
\begin{verbatim}
$ PARAM_PPQ=0.5 ./spinup.sh 4 const 10000 20 hybrid g20km_10ka_hy.nc &> out.g20km_10ka_hy &
\end{verbatim}

Surprisingly perhaps,\footnote{The reason this hybrid run is quicker is fairly deep.  Though the computation of the SSA stress balance is substantially more expensive than the SIA in a per-step sense---and in fact the hybrid model is doing \emph{both} the SIA and SSA computation at each step!---the SSA stress balance in combination with the mass continuity equation causes the maximum diffusivity in the ice sheet to be substantially lower during the run.  (To see this contrast do ``\texttt{ncview ts_g20km_10ka*nc} and view variables \texttt{max_diffusivity} and \texttt{dt}.)  Because the maximum diffusivity controls the time-step in the PISM adaptive time-stepping scheme \cite{BBL}, the number of time steps is so much reduced that the run takes half the time.} this run finishes in half the time of the computation in the previous section, about 10 minutes (0.1649 wall clock hours) on a 2012-era 4-core laptop; as before do
\begin{verbatim}
$ ncdump -h g20km_10ka_hy.nc |grep history
\end{verbatim}
to see the measurement on your machine.  Note that the number reported as ``\texttt{PETSc MFlops}'' from this run is about $2.6 \times 10^5$, many orders of magnitude larger than the previous run.  This is because calls to the PETSc library are used when solving the non-linear and non-local SSA stress balance in parallel.

The results of this run are shown in Figure \ref{fig:secondoutputcoarse}.  We show the basal sliding speed field \texttt{cbase} in this Figure, where Figure \ref{fig:firstoutput} had the \texttt{mask}, but the reader can check that \texttt{cbase}=0 in the nonsliding SIA-only result \texttt{g20km_10ka.nc}.

\begin{figure}[ht]
\centering
\mbox{\includegraphics[height=2.75in,keepaspectratio=true]{g20km-10ka-hy-usurf} \includegraphics[height=2.75in,keepaspectratio=true]{g20km-10ka-hy-csurf} \includegraphics[height=2.75in,keepaspectratio=true]{g20km-10ka-hy-cbase}}
\caption{Fields from output file \texttt{g20km_10ka_hy.nc}.  Left: \texttt{usurf}, the ice sheet surface elevation in meters.  Middle: \texttt{csurf}, the surface speed in m/a, including the 100 m/a contour (solid black).  Right: the sliding speed \texttt{cbase}, shown the same way as \texttt{csurf}.}
\label{fig:secondoutputcoarse}
\end{figure}

The basal sliding velocity, though critical to the response of the ice to changes in climate, is essentially unobservable.  On the other hand, because of relatively-recent advances in radar and image technology and processing \cite{Joughin2002}, the surface velocity of an ice sheet is an observable.  So, how good is our model result for \texttt{csurf}?  Figure \ref{fig:csurfvsobserved} compares the radar-observed \texttt{surfvelmag} field in the downloaded SeaRISE-Greenland data file \texttt{Greenland_5km_v1.1.nc} with the just-computed PISM result.

\begin{figure}[ht]
\centering
\mbox{\includegraphics[height=2.75in,keepaspectratio=true]{Greenland-5km-v1p1-surfvelmag} \includegraphics[height=2.75in,keepaspectratio=true]{g20km-10ka-hy-csurf} \includegraphics[height=2.75in,keepaspectratio=true]{g10km-10ka-hy-csurf}}
\caption{Comparing observed and modeled surface speed.  All figures have a common scale (m/a), with 100 m/a contour shown (solid black).  Left: \texttt{surfvelmag}, the observed values from SeaRISE data file \texttt{Greenland_5km_v1.1.nc}.  Middle: \texttt{csurf} from \texttt{g20km_10ka_hy.nc}.  Right: \texttt{csurf} from \texttt{g10km_10ka_hy.nc}.}
\label{fig:csurfvsobserved}
\end{figure}

The reader might agree with these broad qualitative judgements:
\begin{itemize}
\item the model results and the observed surface velocity look similar, but
\item in the model, southern Greenland has more near-margin fast-flowing ice than in the observed field, which has more localized flow, and
\item the observed Northeast Greenland ice stream is much more distinct than shown in the model.
\end{itemize}

We can compare these easily-generated PISM results to some recent observed-vs-model comparisons of surface velocity maps, of exactly this type, for example Figure 1 in \cite{Priceetal2011} and Figure 8 in \cite{Larouretal2012}.  Note that only ice-sheet-wide parameters and models were used here, that is, each location in the ice sheet was modeled by the same physics.  By comparison, those other results involved tuning a large number of subglacial parameters to values which would yield close match to observations of the surface velocity.  Such tuning techniques are usually called ``inversion'' or ``assimilation'' of the surface velocity data.  Such methods are also possible in PISM if desired; see \cite{vanPeltetal2013} (inversion of DEMs for basal topography) and \cite{Habermannetal2013} (inversion surface velocities for basal shear stress) for PISM-based inversion methods and analysis.

Of course, this is just one model of many ``in'' PISM in the sense that we are free to find and adjust important parameters.  The first parameter change we address is one of the most important: grid resolution.


\subsection{Third run: higher resolution}  \label{subsect:higherresrun}

Now we change one key parameter, the grid resolution.  If you can let it run overnight, do
\begin{verbatim}
$ PARAM_PPQ=0.5 ./spinup.sh 4 const 10000 10 hybrid g10km_10ka_hy.nc &> out.g10km_10ka_hy &
\end{verbatim}
It takes about four hours on a 2012-era 4 core laptop.  Supposing you have an available supercomputer you can change ``\texttt{mpiexec -n 4}'' to ``\texttt{mpiexec -n N}'' where \texttt{N} is a substantially larger number, up to 200 or so.

Some fields from the result \verb|g10km_10ka_hy.nc| are shown in Figure \ref{fig:secondoutputfiner}.  Figure \ref{fig:csurfvsobserved} also compares observed velocity to the model results from 20 km and 10 km grids.  Model results differ even when the only change is the resolution!  Importantly for ice sheet modeling, using higher resolution ``picks up'' more detail in the bed elevation and climate data.

\begin{figure}[ht]
\centering
\mbox{\includegraphics[height=2.75in,keepaspectratio=true]{g10km-10ka-hy-usurf} \includegraphics[height=2.75in,keepaspectratio=true]{g10km-10ka-hy-csurf} \includegraphics[height=2.75in,keepaspectratio=true]{g10km-10ka-hy-cbase}}
\caption{Fields from output file \texttt{g10km_10ka_hy.nc}.  Compare Figure \ref{fig:secondoutputcoarse}, which only differs by resolution.  Left: \texttt{usurf} in meters.  Middle: \texttt{csurf} in m/a.  Right: \texttt{cbase} in m/a.}
\label{fig:secondoutputfiner}
\end{figure}

As a different kind of comparison, Figure \ref{fig:ivolboth} shows ice volume time series \texttt{ivol} for 20km and 10km runs done here.  We see that this result depends on resolution, in particular because higher resolution grids allow the model to better resolve the flux through topographically-controlled outlet glaciers (compare \cite{Pfefferetal2008}).  However, because the total ice sheet volume is a highly-averaged quantity, the \texttt{ivol} difference from 20km and 10km resolution runs is only about one part in 60 (about 1.5\%) at the final time.  By contrast, as is seen in the near-margin ice in various locations shown in Figure \ref{fig:csurfvsobserved}, the ice velocity at a particular location may change by 100\% when the resolution changes from 20km to 10km.

Roughly speaking, the reader should only consider trusting those model results which are demonstrated to be robust across a range of model parameters, and, in particular, which are shown to be relatively-stable among relatively-high resolution results for a particular case.  Using a supercomputer is justifiable merely to confirm that lower-resolution runs were already ``getting'' a given feature!

\begin{figure}[ht]
\centering
\includegraphics[width=4.5in,keepaspectratio=true]{ivol-both-g20km-g10km}
\caption{Time series of modeled ice sheet volume on 20km and 10km grids.  The present-day ice sheet has volume about $2.9\times 10^6\,\text{km}^3$ \cite{BamberLayberryGogenini}, the initial value seen in both runs.}
\label{fig:ivolboth}
\end{figure}


\subsection{Fourth run: paleo-climate model spin-up}  \label{subsect:paleorun}  

A this point we have barely mentioned one of the most important players in the ice sheet modeling game: the surface mass balance (SMB) model.  Specifically, an SMB model combines precipitation (e.g.~\cite{Balesetal2001} for present-day Greenland) and a model for melt based on an approximation of the energy available at the ice surface \cite{Hock05}.  As noted, previous runs in this section used a ``constant-climate'' assumption, which specifically meant using the modeled present-day SMB rates from the regional climate model RACMO \cite{Ettemaetal2009}, as contained in the SeaRISE-Greenland data set \verb|Greenland_5km_v1.1.nc|.  SMB is a key input, of course, because, while a physical model of ice dynamics only describes the rate of movement of the ice, the SMB (and the sub-shelf melt rate) determine changes in the boundary geometry.  That boundary geometry largely determines the stresses seen by the stress balance.

There are other methods for producing SMB than to use present-day modeled values.  We now try such a method, one that does ``paleo-climate spin-up'' for a Greenland ice sheet model.  Of course, direct measurements of prior climates in Greenland are not available as data!  There are, however, estimates of prior surface temperatures at the locations of ice cores \cite[for GRIP]{JohnsenetalGRIP}.  Also there are estimates of prior global sea level \cite{Imbrieetal1984} which can be used to determine where the flotation criterion is applied, which is how PISM's \verb|mask| variable is determined (in part).  Also, models have been constructed for how precipitation differs from the present-day values, based on paleo-records of surface temperature \cite{Huybrechts02}.  Essentially for demonstration purposes, these are all used in the next run.  They are further documented in PISM's Climate Forcing Manual.

But there is more to a paleo-climate model than ``applying'' estimated temperature and precipitation.  One must compute melt in order to compute SMB.  This is done using a temperature-index model, a so-called ``positive degree-day'' (PDD) model \cite{Hock05}; see the Climate Forcing Manual.  Such a model has key parameters for how much snow and/or ice is melted when surface temperatures spend time near or above zero degrees.

To summarize the paleo-climate model applied next, temperature offsets from the GRIP core record affect the snow energy balance, and thus rates of melting and runoff calculated by the PDD model.  In warm periods there is more marginal ablation, but precipitation may also increase (according to a temperature-offset model \cite{Huybrechts02}).  Additionally sea level undergoes changes in time and this affects which ice is floating.  Finally we add an earth deformation model, which responds to changes in ice load by moving the bedrock elevation \cite{BLKfastearth}.

To see how all this translates into PISM options, do
\begin{verbatim}
$ PISM_DO=echo PARAM_PPQ=0.5 REGRIDFILE=g20km_10ka_hy.nc \
  ./spinup.sh 4 paleo 25000 20 hybrid g20km_25ka_paleo.nc
\end{verbatim}
You will see the impressive command (reformatted for clarity):
\small
\begin{verbatim}
mpiexec -n 4 pismr -boot_file pism_Greenland_5km_v1.1.nc -Mx 76 -My 141 \
  -Mz 101 -Mbz 11 -z_spacing equal -Lz 4000 -Lbz 2000 -skip -skip_max 10 \
  -ys -25000 -ye 0 \
  -regrid_file g20km_10ka_hy.nc -regrid_vars litho_temp,thk,enthalpy,tillwat,bmelt \
  -bed_def lc \
  -atmosphere searise_greenland,delta_T,paleo_precip -surface pdd \
  -atmosphere_paleo_precip_file pism_dT.nc -atmosphere_delta_T_file pism_dT.nc \
  -ocean constant,delta_SL -ocean_delta_SL_file pism_dSL.nc \
  -ocean_kill pism_Greenland_5km_v1.1.nc -sia_e 3.0 \
  -ssa_sliding -topg_to_phi 15.0,40.0,-300.0,700.0 -pseudo_plastic -pseudo_plastic_q 0.5 \
  -till_effective_fraction_overburden 0.02 -tauc_slippery_grounding_lines \
  -ts_file ts_g20km_25ka_paleo.nc -ts_times -25000:yearly:0 \
  -extra_file ex_g20km_25ka_paleo.nc -extra_times -25000:100:0 \
  -extra_vars diffusivity,temppabase,tempicethk_basal,bmelt,tillwat,csurf,\
mask,thk,topg,usurf,hardav,cbase,tauc \ 
  -o g20km_25ka_paleo.nc
\end{verbatim}
\normalsize

There are several key changes from the command on page \pageref{firstcommand}.  First, we do not start from scratch but instead from a previously computed near-equilibrium result:
\begin{verbatim}
  -regrid_file g20km_10ka_hy.nc -regrid_vars litho_temp,thk,enthalpy,tillwat,bmelt \
\end{verbatim}
For more on regridding see subsection \ref{sec:regridding}.  Then we turn on the earth deformation model with option \verb|-bed_def lc|; see subsection \ref{subsect:beddef}.  After that the atmosphere and surface (PDD) models are turned on and the files they need are identified:
\begin{verbatim}
  -atmosphere searise_greenland,delta_T,paleo_precip -surface pdd \
  -atmosphere_paleo_precip_file pism_dT.nc -atmosphere_delta_T_file pism_dT.nc \
\end{verbatim}
Then the ocean model, which provides both a subshelf melt rate and a time-dependent sealevel to the ice dynamics core, is turned on with \verb|-ocean constant,delta_SL| and the file it needs is identified with \verb|-ocean_delta_SL_file pism_dSL.nc|.  For all of these ``forcing'' options, see the PISM Climate Forcing Manual.  The remainder of the options are similar or identical to the run that created \verb|g20km_10ka_hy.nc|.

To actually start the run, do:
\begin{verbatim}
$ PARAM_PPQ=0.5 REGRIDFILE=g20km_10ka_hy.nc \
  ./spinup.sh 4 paleo 25000 20 hybrid g20km_25ka_paleo.nc &> out.g20km_25ka_paleo &
\end{verbatim}

FIXME: show results


\subsection{More runs: grid sequencing}  \label{subsect:gridseq}  

% used 20km result to do   10km@1ka --> 5km@200a

\subsection{More runs: a sliding parameter study}  \label{subsect:paramstudy}

% 9 runs: -pseudo_plastic_q   0.0 0.25 1.0   -sia_e   1 3 10


\subsection{Handling NetCDF files}\label{subsect:nctoolsintro}  PISM takes one or more NetCDF files as input, then it does some computation, and then it produces one or more NetCDF files as output.  Usually, other tools help to extract meaning from these NetCDF files, and yet more tools help with creating PISM input files or post-processing PISM output files.

Table \ref{tab:NetCDFview} lists such tools.  We frequently use \texttt{ncview}, ``\texttt{ncdump -h}'', and NCO for quick visualization, metadata examination, and command-line manipulation, respectively.  Visualization tools IDV and PyNGL are especially useful.  

To use CDO on PISM files, first run the script \texttt{nc2cdo.py}, from the \texttt{util/} PISM directory, on the file.  This fixes the metadata so that CDO will understand the mapping.

Regarding the creation of input files for PISM, see the section \ref{sec:bootstrapping-format} and table \ref{tab:modelhierarchy} for ideas about the data necessary for modeling.

\newcommand{\netcdftool}[1]{#1\index{NetCDF!tools!#1}}
\begin{table}[ht]
\centering
\small
\begin{tabular}{llp{0.45\linewidth}}
  \toprule
  \textbf{Tool} & \textbf{Site} & \textbf{Function} \\
  \midrule
  & \url{www.unidata.ucar.edu/software/netcdf/} & root for NetCDF information \\
  \midrule
  \netcdftool{\texttt{ncdump}} & \emph{included with any NetCDF distribution} & dump binary NetCDF as \texttt{.cdl} (text) file \\
  \netcdftool{\texttt{ncgen}} & \emph{included with any NetCDF distribution} & convert \texttt{.cdl} file to binary NetCDF \\
  \midrule
  \netcdftool{CDO} & \url{http://code.zmaw.de/projects/cdo} & = Climate Data Operators; command-line tools, including conservative re-mapping \\
  \netcdftool{IDV} & \url{http://www.unidata.ucar.edu/software/idv/} & more complete visualization \\
  \netcdftool{NCO}\index{NCO (NetCDF Operators)} & \url{http://nco.sourceforge.net/} & = NetCDF Operators; command-line tools\\
  \netcdftool{NCL} &  \url{http://www.ncl.ucar.edu} & = NCAR Command Language\\
  \netcdftool{\texttt{ncview}} & \href{http://meteora.ucsd.edu/~pierce/ncview_home_page.html}{\texttt{meteora.ucsd.edu/$\sim$pierce}} & quick graphical view \\
  \netcdftool{PyNGL} &  \url{http://www.pyngl.ucar.edu} & Python version of NCL\\
  \bottomrule
\end{tabular}
\normalsize
\caption{A selection of tools for viewing and modifying NetCDF files.}
\label{tab:NetCDFview}
\end{table}


%%% Local Variables: 
%%% mode: latex
%%% TeX-master: "manual"
%%% End: 


% LocalWords:  metadata SPECMAP paleo html IDV
