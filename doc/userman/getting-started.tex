
\section{Getting started}\label{sec:start}\index{PISM!getting started}

\subsection{On installing PISM}

To install PISM, see the \,\emph{Getting PISM}\, tab at \href{http://www.pism-docs.org}{\texttt{www.pism-docs.org}}.  Or use the PISM Installation Manual (PDF).\footnote{At \url{http://www.pism-docs.org/wiki/lib/exe/fetch.php?media=installation.pdf}.}

Once PISM is installed, the executable \texttt{pismr} should be available on your system's ``path''; confirm this with ``\texttt{which pismr}''.  The instructions below assume you are using a \texttt{bash} shell (or one that accepts \texttt{bash} syntax).  They also assume you have the PISM source code in the directory ``\texttt{pism/}''.

\subsection{A Greenland ice sheet example}

We get started with an extended example showing how to generate initial states for prognostic model experiments on the Greenland ice sheet.  Ice sheet and glacier model studies often involve modeling present and past states using actions like the ones demonstrated in this section.  The particular choices made here are motivated by the evaluation of initialization methods in \cite{AschwandenAdalgeirsdottirKhroulev}.

We use data assembled by the \href{http://websrv.cs.umt.edu/isis/index.php/SeaRISE_Assessment}{Sea-level Response to Ice Sheet Evolution (SeaRISE)} group \cite{Bindschadler2013SeaRISE}.  SeaRISE was a community-organized assessment process providing an upper bound on ice sheet contributions to sea level in the next 100--200 years, especially for the IPCC AR5 report in 2013.

This example is a hands-on first look at PISM.  It is not an in-depth tutorial, and some details of what is happening will only be explained later.  The remainder of this manual lists PISM options, discusses nontrivial modeling choices, and explains the ways users may need to preprocess input data.

Some of the PISM output figures in this section were produced using a supercomputer.  Though the basic run here, on a rather coarse $20\,\textrm{km}$ grid, can be done on a typical workstation or laptop, PISM is designed to make high resolution possible by exploiting large-scale parallel processing \cite[among many other examples]{AschwandenAdalgeirsdottirKhroulev,Golledgeetal2012,Golledgeetal2013}.


\subsection{Input data}

The NetCDF data used to initialize SeaRISE runs is freely-available online: 
\medskip

\centerline{\protect{\textbf{\url{http://websrv.cs.umt.edu/isis/index.php/Present_Day_Greenland}}}}
\medskip

\noindent The quickest way to get the file is to do
\begin{verbatim}
$ cd pism/examples/std-greenland
$ ./preprocess.sh
\end{verbatim}
\noindent The script \texttt{preprocess.sh} requires \texttt{wget} and also the NetCDF Operators (``NCO''; \url{http://nco.sourceforge.net/}).  It downloads the version 1.1 of the SeaRISE ``master'' present-day data set, which contains ice thickness and bedrock topography from BEDMAP \cite{BamberLayberryGogenini}, and modeled precipitation rates from RACMO \cite{Ettemaetal2009}, among other fields.

In particular, it creates three new NetCDF files which can be read by PISM.  The spatially-varying fields from the downloaded ``master'' file, with adjusted metadata, go in the PISM-readable file \texttt{pism_Greenland_5km_v1.1.nc}.  The other two new files contain famous time-dependent paleo-climate records from ice core and seabed core records; \texttt{pism_dT.nc} has GRIP \cite{JohnsenetalGRIP} while \texttt{pism_dSL.nc} has SPECMAP \cite{Imbrieetal1984}.

Any of these NetCDF files can be viewed with \texttt{ncview} or other NetCDF visualization tools; see Table \ref{tab:NetCDFview} below.  An application of IDV to the master data set produced Figure \ref{fig:sr-input}, for example.  Use \texttt{ncdump -h} to see the metadata and history of the files.

\begin{figure}[ht]
\centering
\mbox{\includegraphics[width=2.0in,keepaspectratio=true]{sr-greenland-thk}
  \qquad
  \includegraphics[width=2.0in,keepaspectratio=true]{sr-greenland-topg}
  \qquad
  \includegraphics[width=2.0in,keepaspectratio=true]{sr-greenland-prcp}}
\caption{The input present-day ice thickness (left; m), bedrock elevation (center; m), and present-day precipitation (right; m $\text{a}^{-1}$ ice equivalent) for SeaRISE-Greenland.  These are fields \texttt{thk}, \texttt{topg}, and \texttt{precipitation}, respectively, in file \texttt{pism_Greenland_5km_v1.1.nc} produced by running \texttt{preprocess.sh}.}
\label{fig:sr-input}
\end{figure}


\subsection{First run}   \label{subsect:runscript}  Like many Unix programs, PISM allows many command-line options.  Because PISM handles a wide variety of ice sheet, shelf, and glacier sub-models and configurations, the list of possible command-line options is long; see sections \ref{sec:boot} through \ref{sec:practical-usage} of this manual.  Therefore, in practice, one often builds scripts to run PISM with the correct options.

Our script is called ``\texttt{spinup.sh}''.  Initializing ice sheets can be done by computing approximate steady states with constant boundary data, or, in some cases, by integrating paleo-climatic and long-time-scale information to build a model for the present state of the ice sheet.  Both of these possibilities are illustrated in the \texttt{spinup.sh} script.  The spin-up stage may even require the most processor-hours, compared to a follow-on ``experiment'' or ``forecast'' stages.

We are ready to run PISM.  To see what can be done with the script, read the usage message it produces:
\begin{verbatim}
$ ./spinup.sh
\end{verbatim}

The simplest spin-up approach is to use a ``constant-climate'' model.  We take this approach first.  To see a more detailed view of the run we will do first, try:
\begin{verbatim}
$ PISM_DO=echo ./spinup.sh 4 const 10000 20 sia g20km_10ka.nc
\end{verbatim}
Setting the environment variable \texttt{PISM_DO} in this way tells \texttt{spinup.sh} just to print out the commands it is about to run, not do them.  The ``proposed'' run looks like this:
\small
\begin{verbatim}
mpiexec -n 4 pismr -boot_file pism_Greenland_5km_v1.1.nc -Mx 76 -My 141 \
  -Mz 101 -Mbz 11 -z_spacing equal -Lz 4000 -Lbz 2000 -skip -skip_max 10 \
  -ys -10000 -ye 0 -surface given -surface_given_file pism_Greenland_5km_v1.1.nc \
  -ocean_kill pism_Greenland_5km_v1.1.nc -sia_e 3.0 \
  -ts_file ts_g20km_10ka.nc -ts_times -10000:yearly:0 \
  -extra_file ex_g20km_10ka.nc -extra_times -10000:100:0 \
  -extra_vars diffusivity,temppabase,tempicethk_basal,bmelt,tillwat,csurf,mask,thk,topg,usurf \
  -o g20km_10ka.nc
\end{verbatim}
\normalsize
Let's briefly deconstruct this run.

At the front is ``\texttt{mpiexec -n 4 pismr}''.  This means that the PISM executable \texttt{pismr} is run in parallel on four processes (e.g.~cores).  Though this example assumes you have a workstation or laptop with at least 4 cores, it will work with 1 to 100 processors, with reasonably good scaling in speed.  (Scaling can be good with far more processors if we run at higher spatial resolution.)  The executable name ``\texttt{pismr}'' stands for the standard ``run'' mode of PISM, in contrast to other, specialized modes described later (e.g.~in sections \ref{sec:verif} and \ref{sec:simp}).

Next, the proposed run uses option \texttt{-boot_file} to start the run by ``bootstrapping,'' a term describes the creation, by heuristics and highly-simplified models, of the complete initial conditions required for a deterministic, time-dependent ice dynamics model.  Then the options describe a $76\times 141$ point grid in the horizontal, which gives 20 km grid spacing in both directions.  Then there are choices about the vertical extent and resolution of the computational grid; more on those later.  After that we see a description of the time-axis, with a start and end time given: ``\texttt{-ys -10000 -ye 0}''.

Then we get the instructions that tell PISM to read the upper surface boundary conditions (i.e.~climate) from a file: ``\texttt{-surface given -surface_given_file pism_Greenland_5km_v1.1.nc}''.  For more on these choices, see subsection \ref{sec:climate-inputs}, and also the PDF PISM Climate Forcing Manual.

Following that there are longish options describing the fields we want as output, including scalar time series (``\texttt{-ts_file ts_g20km_10ka.nc -ts_times -10000:yearly:0}''; see section \ref{sec:practical-usage}), time- and space-dependent fields (``\texttt{-extra_file ...}''; again see section \ref{sec:practical-usage}), and finally the named output file (``\texttt{-o g20km_10ka.nc}'').

Now let's actually get the run going:
\begin{verbatim}
$ ./spinup.sh 4 const 10000 20 sia g20km_10ka.nc &> out.g20km_10ka &
\end{verbatim}
\noindent The terminating ``\verb|&|'' asks unix to run the command in the background, so we can keep working in the current shell.  Because we have re-directed the text output, PISM will show what it is doing in the text file \texttt{out.g20km_10ka}.  Using \texttt{less} is a good way to watch such a growing text-output file.


\subsection{Watching the first run}  \label{subsect:watchrun}  The next paragraphs describe what happens and what files are produced by the run which is underway.  The modeling choices represented here are reasonable, but they are not the only way to do it.  The user is encouraged to experiment; that is the point of a model!

As soon as the run starts it creates time-dependent NetCDF files \texttt{ts_g20km_10ka.nc} and \texttt{ex_g20km_10ka.nc}.  Actually the latter file spatially-dependent file is created after the first 100 years of model run, but that is a few seconds in this case.  In fact, the command \texttt{-extra_file ex_g20km_10ka.nc -extra_times -10000:100:0} will add a spatially-dependent ``frame'' at model times -9900, -9800, \dots, 0.  One way to get a snapshot of how many frames this file contains at any time is to do
\begin{verbatim}
$ ncdump -h ex_g20km_10ka.nc |grep "time ="
\end{verbatim}
but a more interactive way is to look at the output graphically with
\begin{verbatim}
$ ncview ex_g20km_10ka.nc
\end{verbatim}
In any case, we see that \texttt{ex_g20km_10ka.nc} contains ``movies'' of the fields chosen by the option \texttt{-extra_vars diffusivity,...,thk,topg,usurf}.  A frame of the ice thickness \texttt{thk} is shown in Figure \ref{fig:growing} (left).

The time-series file \texttt{ts_g20km_10ka.nc} is also growing.  It contains spatially-averaged ``scalar'' diagnostics like the total ice volume or the ice-sheet-wide maximum velocity (variable \texttt{ivol} and \texttt{max_hor_vel}, respectively).  It can be viewed
\begin{verbatim}
$ ncview ts_g20km_10ka.nc
\end{verbatim}
Generally PISM ``buffers'' its time series, so the file may only be updated when the buffer is full (see section \ref{sec:saving-time-series}), but, as the buffer is also cleared everytime a frame is written to an \texttt{-extra_file}, in this case the file \texttt{ts_g20km_10ka.nc} is updated frequently.  The growing time series for \texttt{ivol} is shown in Figure \ref{fig:growing} (right).  After such inspections of the diagnostic files \texttt{ex_g20km_10ka.nc} and \texttt{ts_g20km_10ka.nc}, the reader will agree that PISM can produce quite a bit of information about a run, but in a user-controllable manner.

\begin{figure}[ht]
\centering
\mbox{\includegraphics[height=5.0in,keepaspectratio=true]{ex-growing-thk-g20km}
  \qquad \includegraphics[height=5.0in,keepaspectratio=true]{ts-growing-ivol-g20km}}
\caption{Two views produced by \texttt{ncview} during a PISM model run for the Greenland ice sheet.  Left: Field \texttt{thk}, the ice sheet thickness, a space-dependent frame from file \texttt{ex_g20km_10ka.nc}.  Right: Time-series \texttt{ivol}, the total ice sheet volume, from file \texttt{ts_g20km_10ka.nc}. }
\label{fig:growing}
\end{figure}

At the end of the run the output file \texttt{g20km_10ka.nc} is generated.  To see a report on computational  performance, we do:
\begin{verbatim}
$ ncdump -h g20km_10ka.nc |grep history
    :history = "user@machine 2013-11-23 15:57:22 AKST: PISM done.  Performance stats:
0.3435 wall clock hours, 1.3738 proc.-hours, 7274.0065 model years per proc.-hour,
PETSc MFlops = 0.03.\n",
\end{verbatim}
The whole first run took about 21 minutes (0.3435 wall clock hours) on a 2012-era 4-core laptop.

The intention of this run was to generate a minimal model of the Greenland ice sheet in approximate steady-state with a steady (constant-in-time) climate.  Figure \ref{fig:firstoutput} shows some fields from \texttt{g20km_10ka.nc}.  In the next subsection  we will start discussing their ``quality'' as model results.

\begin{figure}[ht]
\centering
\mbox{\includegraphics[height=2.75in,keepaspectratio=true]{g20km-10ka-usurf} \includegraphics[height=2.75in,keepaspectratio=true]{g20km-10ka-csurf} \includegraphics[height=2.75in,keepaspectratio=true]{g20km-10ka-mask}}
\caption{Fields from output file \texttt{g20km_10ka.nc}.  Left: \texttt{usurf}, the ice sheet surface elevation in meters.  Middle: \texttt{csurf}, the surface velocity in meters/year ($=$ m/a), including the 100 m/a contour (solid black).  Right: \texttt{mask}, with 0 = ice-free land, 2 = ice, 4 = ice-free ocean.}
\label{fig:firstoutput}
\end{figure}


\subsection{Second run: a better ice-dynamics model}  \label{subsect:ssarun}  

It is widely-understood that ice sheets slide on their bases, especially when liquid water is present at the base (see \cite{Joughinetal2001,MacAyeal}, among many others).  An important aspect of modeling such sliding is the inclusion of membrane or ``longitudinal'' stresses into the stress balance \cite{BBssasliding}.  The basic stress balance in PISM which uses membrane stresses is the Shallow Shelf Approximation (SSA) \cite{WeisGreveHutter}.  The stress balance used in the previous section was, by contrast, the (thermomechanically-coupled) non-sliding, non-membrane-stress Shallow Ice Approximation (SIA) \cite{BBL,EISMINT00}.  The preferred ice dynamics model within PISM, that allows both sliding balanced by membrane stresses and shear flow as described by the SIA, is the SIA+SSA ``hybrid'' model \cite{BBssasliding,Winkelmannetal2011}.  For more on such theory see section \ref{sec:dynamics} of the current \emph{Manual}.

The practical disadvantage of models of sliding is that a distinctly-uncertain parameter space must be introduced.  This especially involves parameters controlling the amount and pressure of subglacial water (see \cite{AschwandenAdalgeirsdottirKhroulev,Clarke05,Tulaczyketal2000,vanPeltOerlemans2012} among other references).  In this regard, PISM uses the concept of a saturated and pressurized subglacial till with a modeled distribution of yield stress  \cite{BBssasliding,SchoofStream}.  The yield stress arises from the PISM model of the production of subglacial water, which is itself computed through the conservation of energy model \cite{AschwandenBuelerKhroulevBlatter}.

We use such models in the rest of this Getting Started section.  We will show some steps which analyze the difference between observed and modeled surface velocity of the Greenland ice sheet.  While the \texttt{spinup.sh} script has default sliding parameters, for demonstration purposes we change one parameter; we replace the default power $q=0.25$ in the sliding law, an equation which relates both the subglacial sliding velocity and the till yield stress to the basal shear stress which appears in the SSA stress balance, by a less ``plastic'' choice $q=0.5$.  (See subsection \ref{subsect:basestrength} for more on this important parameter, and other related parameters.)

To see the run we propose, do
\begin{verbatim}
$ PISM_DO=echo PARAM_PPQ=0.5 ./spinup.sh 4 const 10000 20 hybrid g20km_10ka_hy.nc
\end{verbatim}
Now remove the setting of \texttt{PISM_DO} and redirect the text output into a file, and start the run:
\begin{verbatim}
$ PARAM_PPQ=0.5 ./spinup.sh 4 const 10000 20 hybrid g20km_10ka_hy.nc &> out.g20km_10ka_hy &
\end{verbatim}

Surprisingly perhaps,\footnote{The reason this hybrid run is quicker is fairly deep.  Though the computation of the SSA stress balance is substantially more expensive than the SIA in a per-step sense---and in fact the hybrid model is doing both the SIA and SSA computation at each step!---the SSA stress balance in combination with the mass continuity equation causes the maximum diffusivity in the ice sheet to be substantially lower during the run.  Because the maximum diffusivity controls the time-step in the PISM adaptive time-stepping scheme \cite{BBL}, the number of time steps is so much reduced that the run takes half the time.  (To see this contrast do ``\texttt{ncview ts_g20km_10ka*nc} and view variables \texttt{max_diffusivity} and \texttt{dt}.) } this finishes in half the time of the computation in the previous section, about 10 minutes (0.1649 wall clock hours) on a 2012-era 4-core laptop:
\begin{verbatim}
$ ncdump -h g20km_10ka_hy.nc |grep history
    :history = "user@machine 2013-11-23 17:36:55 AKST: PISM done.  Performance stats:
0.1649 wall clock hours, 0.6598 proc.-hours, 15146.2867 model years per proc.-hour,
PETSc MFlops = 264118.55.\n",
\end{verbatim}
Note that the number reported as ``\texttt{PETSc MFlops}'' is many orders of magnitude larger because calls to the PETSc library are used when solving the non-local SSA stress balance.

The results of this run are shown in Figure \ref{fig:secondoutputcoarse}.  We show the basal sliding speed field \texttt{cbase} in this Figure.  (Replacing \texttt{mask} in Figure \ref{fig:firstoutput}; the reader can check that \texttt{cbase}=0 in the SIA-only result \texttt{g20km_10ka.nc}.)

\begin{figure}[ht]
\centering
\mbox{\includegraphics[height=2.75in,keepaspectratio=true]{g20km-10ka-hy-usurf} \includegraphics[height=2.75in,keepaspectratio=true]{g20km-10ka-hy-csurf} \includegraphics[height=2.75in,keepaspectratio=true]{g20km-10ka-hy-cbase}}
\caption{Fields from output file \texttt{g20km_10ka_hy.nc}.  Left: \texttt{usurf}, the ice sheet surface elevation in meters.  Middle: \texttt{csurf}, the surface velocity in m/a, including the 100 m/a contour (solid black).  Right: \texttt{cbase}, shown the same way as \texttt{csurf}.}
\label{fig:secondoutputcoarse}
\end{figure}

The basal sliding velocity, though critical to the response of the ice to changes in climate, is essentially unobservable.  On the other hand, at the present state of technology the surface velocity of an ice sheet is an observable.\footnote{This is because of relatively-recent advances in radar and image technology and processing \cite{Joughin2002}.}  So, how good is our model result for \texttt{csurf}?  The left-hand two parts of Figure \ref{fig:csurfvsobserved} compare the \texttt{surfvelmag} field in the downloaded SeaRISE-Greenland data file \texttt{Greenland_5km_v1.1.nc} with the just-computed PISM result.  The reader might agree with these broad qualitative judgements:
\begin{itemize}
\item the model results and the observed surface velocity look similar, but
\item in the model, southern Greenland has fast-flowing ice at the margin, while the observations show much more localized flow, and
\item the observed Northeast Greenland ice stream is much more distinct than shown in the model.
\end{itemize}
Of course, this is just one model of many ``in'' PISM.  We are free to find and adjust important parameters.  The first we address is one of the most important: grid resolution.

We can compare these easily-generated PISM results to some recent observed-vs-model comparisons of surface velocity maps, of exactly this type, for example Figure 1 in \cite{Priceetal2011} and Figure 8 in \cite{Larouretal2012}.  Note that only ice-sheet-wide parameters and models were used here, that is, each location in the ice sheet was modeled by the same physics.  By comparison, those other results involved tuning a large number of subglacial parameters to values which would yield close match to observations of the surface velocity.  Such tuning techniques are usually called ``inversion'' or ``assimilation'' of the surface velocity data.  Such methods are also possible in PISM if desired; see \cite{vanPeltetal2013} (inversion of DEMs for basal topography) and \cite{Habermannetal2013} (inversion surface velocities for basal shear stress) for PISM-based inversion methods and analysis.

Now we change one key parameter, the grid resolution.  If you can let it run overnight on a laptop,\footnote{Or change ``\texttt{4}'', the number of processes in the call to \texttt{spinup.sh}, to a substantially larger number if you have an available supercomputer.} do
\begin{verbatim}
$ PARAM_PPQ=0.5 ./spinup.sh 4 const 10000 10 hybrid g10km_10ka_hy.nc &> out.g10km_10ka_hy &
\end{verbatim}
This run takes about four hours on a 2012-era 4 core laptop.  We see some fields from the result in Figure \ref{fig:secondoutputfiner}.

\begin{figure}[ht]
\centering
\mbox{\includegraphics[height=2.75in,keepaspectratio=true]{g10km-10ka-hy-usurf} \includegraphics[height=2.75in,keepaspectratio=true]{g10km-10ka-hy-csurf} \includegraphics[height=2.75in,keepaspectratio=true]{g10km-10ka-hy-cbase}}
\caption{Fields from output file \texttt{g10km_10ka_hy.nc}.  Left: \texttt{usurf} in meters.  Middle: \texttt{csurf} in m/a.  Right: \texttt{cbase} in m/a.}
\label{fig:secondoutputfiner}
\end{figure}

Figure \ref{fig:csurfvsobserved} compares observed velocity to the model results from 20 km and 10 km grids.  Model results differ, of course, even when the only change is the resolution.  Importantly for ice sheet modeling, higher resolution model runs ``pick up'' higher resolution bed elevation and climate data.

\begin{figure}[ht]
\centering
\mbox{\includegraphics[height=2.75in,keepaspectratio=true]{Greenland-5km-v1p1-surfvelmag} \includegraphics[height=2.75in,keepaspectratio=true]{g20km-10ka-hy-csurf} \includegraphics[height=2.75in,keepaspectratio=true]{g10km-10ka-hy-csurf}}
\caption{Comparing modeled surface speed to observations.  All are in m/a with 100 m/a contour solid black.  Left: \texttt{surfvelmag}, the observed values from SeaRISE data file \texttt{Greenland_5km_v1.1.nc}.  Middle: \texttt{csurf} from \texttt{g20km_10ka_hy.nc}.  Right: \texttt{csurf} from \texttt{g10km_10ka_hy.nc}.}
\label{fig:csurfvsobserved}
\end{figure}

As a different kind of comparison, Figure \ref{fig:ivolboth} shows ice volume time series \texttt{ivol} for 20km and 10km runs done here.  We see that this result depends on resolution, in particular because higher resolution grids allow the model to better resolve the flux through topographically-controlled outlet glaciers (compare \cite{Pfefferetal2008}).  However, because the total ice sheet volume is a highly-average quantity, the \texttt{ivol} difference from 20km and 10km resolution runs is only about one part in 60 (about 1.5\%) at the final time.  By contrast, as is seen in the near-margin ice in various locations shown in Figure \ref{fig:csurfvsobserved}, the ice velocity at a particular location may change by 100\% when the resolution changes from 20km to 10km.  Roughly speaking, the reader should only trust model results which are demonstrated to be robust across a range of model parameters, and, in particular, which are shown to be relatively-stable among relatively-high resolution results for a particular case.  Using a supercomputer is, in fact, often justified merely to confirm that lower-resolution runs were already ``getting'' a given feature in model results.

\begin{figure}[ht]
\centering
\includegraphics[width=4.5in,keepaspectratio=true]{ivol-both-g20km-g10km}
\caption{Time series of modeled ice sheet volume on 20km and 10km grids.  The present-day ice sheet has volume approximately $2.91\times 10^6\,\text{km}^3$, the initial value in both runs.}
\label{fig:ivolboth}
\end{figure}


\subsection{Third run: paleo-climate model spin-up}  \label{subsect:paleorun}  


\begin{figure}[ht]
\centering
%  temppabase from last time in ex_g10km_steady.nc and driving stress taud from g10km_SIA.nc
%\mbox{\includegraphics[width=2.0in,keepaspectratio=true]{temppabase}
%  \qquad \includegraphics[width=2.1in,keepaspectratio=true]{taud}}
\caption{FIXME: Part of the model state at the beginning of paleo-climate-modeling ice sheet spin-up, in file \texttt{g20km_steady.nc} from running \texttt{spinup.sh}.  Left: pressure-adjusted basal temperature ($\phantom{|}^\circ$C; field \texttt{temp_pa}).  Right: driving stress magnitude $\rho g H |\grad h|$ (Pa; field \texttt{taud_mag}).}
\label{fig:sr-spinstart}
\end{figure}


\begin{figure}[ht]
\centering
%  thk, cbase, csurf from g10km_0.nc
%\mbox{\includegraphics[width=2.in,keepaspectratio=true]{thk}
%  \qquad \includegraphics[width=2.in,keepaspectratio=true]{cbase}
%  \qquad \includegraphics[width=2.in,keepaspectratio=true]{csurf}}
\caption{FIXME Model for the present-day Greenland ice sheet, based on spin-up over 125,000 model years using paleo-climate forcing.  Left: ice thickness (m).  Center: basal sliding speed (m/a).  Right: surface speed (m/a).  These are fields \texttt{thk}, \texttt{cbase}, and \texttt{csurf} from file \texttt{g10km_0.nc}.}
\label{fig:sr-spindone-map}
\end{figure}

FIXME The actual paleo-climate-driven spin-up starts from model state \texttt{g20km_steady.nc}.  We turn on three new mechanisms, climatic forcing, bed deformation, and improved stress balance.  There are two forms of climatic forcing. First, temperature offsets from GRIP core data affect the snow energy balance and thus rates of melting and runoff; the model is a simple temperature-index scheme described at the SeaRISE website.  Thereby the surface mass balance varies over time; in warm periods there is more marginal ablation.  Additionally, sea levels from the SPECMAP cores affects which ice is floating.

Significantly, we start using a more complete ice dynamics model with sliding controlled by a membrane stress balance, the SIA+SSA hybrid model described in section \ref{sec:dynamics}.  This is turned on with option ``\texttt{-ssa_sliding}''.

Also, we turn on a bed deformation model; the option is ``\texttt{-bed_def lc}''.  

These options are discussed in section \ref{sec:modeling-dynamics}.



\subsection{More runs: grid sequencing}  \label{subsect:gridseq}  

\subsection{More runs: a sliding parameter study}  \label{subsect:paramstudy}

% 9 runs:
-pseudo_plastic_q   0.0 0.25 1.0
-sia_e   1 3 10

\subsection{Going forward}  \label{subsect:forecastcaution}  Because real ice sheets, and ice sheet models too, have a ``memory'' of past climates, results of ``forecast'' runs may strongly depend on the nature of the spin-up process which create the initial conditions.  It is critical to evaluate the quality of the spunup state, for example using present-day observations of surface velocity and other fields \cite{AschwandenAdalgeirsdottirKhroulev}.  Critical thinking about a broad range of modeling hypotheses is prerequisite to building models of future behavior.


\subsection{Handling NetCDF files}\label{subsect:nctoolsintro}  PISM takes one or more NetCDF files as input, then it does some computation, and then it produces one or more NetCDF files as output.  Usually, other tools help to extract meaning from these NetCDF files, and yet more tools help with creating PISM input files or post-processing PISM output files.

Table \ref{tab:NetCDFview} lists such tools.  We frequently use \texttt{ncview}, ``\texttt{ncdump -h}'', and NCO for quick visualization, metadata examination, and command-line manipulation, respectively.  Visualization tools IDV and PyNGL are especially useful.  

To use CDO on PISM files, first run the script \texttt{nc2cdo.py}, from the \texttt{util/} PISM directory, on the file.  This fixes the metadata so that CDO will understand the mapping.

Regarding the creation of input files for PISM, see the section \ref{sec:bootstrapping-format} and table \ref{tab:modelhierarchy} for ideas about the data necessary for modeling.

\newcommand{\netcdftool}[1]{#1\index{NetCDF!tools!#1}}
\begin{table}[ht]
\centering
\small
\begin{tabular}{llp{0.45\linewidth}}
  \toprule
  \textbf{Tool} & \textbf{Site} & \textbf{Function} \\
  \midrule
  & \url{www.unidata.ucar.edu/software/netcdf/} & root for NetCDF information \\
  \midrule
  \netcdftool{\texttt{ncdump}} & \emph{included with any NetCDF distribution} & dump binary NetCDF as \texttt{.cdl} (text) file \\
  \netcdftool{\texttt{ncgen}} & \emph{included with any NetCDF distribution} & convert \texttt{.cdl} file to binary NetCDF \\
  \midrule
  \netcdftool{CDO} & \url{http://code.zmaw.de/projects/cdo} & = Climate Data Operators; command-line tools, including conservative re-mapping \\
  \netcdftool{IDV} & \url{http://www.unidata.ucar.edu/software/idv/} & more complete visualization \\
  \netcdftool{NCO}\index{NCO (NetCDF Operators)} & \url{http://nco.sourceforge.net/} & = NetCDF Operators; command-line tools\\
  \netcdftool{NCL} &  \url{http://www.ncl.ucar.edu} & = NCAR Command Language\\
  \netcdftool{\texttt{ncview}} & \href{http://meteora.ucsd.edu/~pierce/ncview_home_page.html}{\texttt{meteora.ucsd.edu/$\sim$pierce}} & quick graphical view \\
  \netcdftool{PyNGL} &  \url{http://www.pyngl.ucar.edu} & Python version of NCL\\
  \bottomrule
\end{tabular}
\normalsize
\caption{A selection of tools for viewing and modifying NetCDF files.}
\label{tab:NetCDFview}
\end{table}


%%% Local Variables: 
%%% mode: latex
%%% TeX-master: "manual"
%%% End: 


% LocalWords:  metadata SPECMAP paleo html IDV
