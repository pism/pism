
\section{Practical usage}
\label{sec:practical-usage}

\subsection{Input and output}
\label{sec:input-output}
\optsection{Input and output}
\optseealso{Bootstrapping}
\optseealso{Regridding}

PISM is a program that reads NetCDF files and then outputs NetCDF files.  Table \ref{tab:input-output-options} summarizes command-line options controlling the most basic input and output of NetCDF files relating to starting and ending PISM runs.

\begin{table}[ht]
  \centering
 \begin{tabular}{lp{0.6\linewidth}}
    \toprule
    \textbf{Option} & \textbf{Description} \\
    \midrule
    \fileopt{i} & Chooses a PISM output file to restart from.  (``Restart'' because normally the first PISM run must come from data by bootstrapping (\intextoption{boot_file}) or using executables \texttt{pisms} or \texttt{pismv}.) \\
    \fileopt{o} & Chooses the output file name.  Default name is \texttt{unnamed.nc} under \texttt{pismr}, \texttt{simp_exper.nc} under \texttt{pisms}, and \texttt{verify.nc} under \texttt{pismv}.\\
    \txtopt{o_size}{[small, medium, big]} & Chooses the ``size'' of the output file to produce.
    Possible options are \texttt{none} (\emph{no} output file at all), \texttt{small} (only variables necessary to restart
    PISM), \texttt{medium} (the default, includes some diagnostic quantities)
    and \texttt{big} (writes all the variables mentioned in tables~\ref{tab:three-d-diagnostics},~\ref{tab:two-d-diagnostics-1},~\ref{tab:two-d-diagnostics-2},~and~\ref{tab:two-d-diagnostics-3}).  Configuration variables \texttt{output_medium} and \texttt{output_big} list the written variables for those sizes. \\
    \intextoption{dontreadSSAvels} & Turns off reading the \texttt{ubar_ssa}
    and \texttt{vbar_ssa} velocities saved by a previous
    \texttt{-ssa_floating_only} or \texttt{-ssa_sliding} run. \\
    \bottomrule
 \end{tabular}
\caption{Basic NetCDF input and output options}
\label{tab:input-output-options}
\end{table}

Table \ref{tab:stdout} gives a summary of PISM's options controlling
what is printed to the standard output.  This includes the \texttt{-help} and \texttt{-usage} options for getting help at the command line.

\begin{table}[ht]
  \centering
 \begin{tabular}{lp{0.7\linewidth}}
    \toprule
    \textbf{Option} & \textbf{Description} \\
    \midrule
    \intextoption{help} & Gives PISM help message and then a brief description of many PISM and PETSc options.  The run occurs as usual according to the other options, while the option documentation does not get listed if the run didn't get started properly.\\
    \intextoption{info} & Gives almost-always excessive information about PETSc operations during run.  Option \texttt{-verbose} is generally more useful for users. \\
    \intextoption{log_summary}  & At the end of the run gives a performance
    summary and also a synopsis of the PETSc configuration in use.\\
   \intextoption{pismversion} &   Show version number of PISM.\\
   \intextoption{options_left} & Shows an options table which will indicate if
   a user option was not read or was misspelled.\\
   \intextoption{usage} &   Short summary of PISM executable usage, without listing all the options, and without doing the run.\\
   \intextoption{verbose} & Increased verbosity of standard output.  Usually given with an integer level; 0,1,2,3,4,5 are allowed.  At the extremes, \texttt{-verbose 0} produces no stdout at all, \texttt{-verbose 1} prints only warnings and a few high priority messages, and \texttt{-verbose 5} spews a lot of usually-undesirable stuff.  \texttt{-verbose 3} adds a bit more about initialization, especially.  If given without argument (``\texttt{-verbose}'') then sets level 3, while ``\texttt{-verbose 2}'' is the default.\\
   \intextoption{version} &   Show version numbers of PETSc and of PISM.\\
   \bottomrule
  \end{tabular}
\caption{Options controlling PISM's standard output}
\label{tab:stdout}
\end{table}


The following sections describe more input and output options, especially related to saving quantities during a run, or adding to the ``diagnostic'' outputs of PISM.

\subsubsection{PISM file I/O performance}
\label{sec:pism-io-performance}

When working with fine grids\footnote{Resolutions of 2km and higher on the
  whole-Greenland scale is an example.}, the time PISM spends writing output,
spatial time-series or backup files can become significant.

It turns out that it is a lot faster to read and write files using the
\texttt{t,x,y,z} storage order, as opposed to the more convenient
\texttt{t,z,y,x} order.

The reason is that PISM uses the \texttt{x,y,z} order internally\footnote{This
  is not likely to change.} and writing an array in a different order is an
inherently expensive operation.

You can choose one of the three supported output orders using the
\intextoption{o_order} option with one of \texttt{xyz}, \texttt{yxz}, and
\texttt{zyx} as the argument.

To transpose dimensions in an existing file, use the \texttt{ncpdq} (``permute
dimensions quickly'') tool from the NCO (\emph{NetCDF Operators}) suite.

Run
\begin{verbatim}
$ ncpdq -a t,z,zb,y,x bad.nc good.nc
\end{verbatim}%$
to turn \texttt{bad.nc} (with an unspecified inconvenient storage order) into
\texttt{good.nc} using the \texttt{t,z,y,x} order.

PISM also supports NetCDF-4 parallel I/O, which gives better performance in
high-resolution runs and avoids NetCDF-3 file format limitations. (In a
NetCDF-3 file a variable record cannot exceed 4 gigabytes.) Build PISM with
parallel NetCDF-4 and use \txtopt{o_format}{\texttt{netcdf4_parallel}} to
enable this code.

In addition to \texttt{-o_format netcdf4_parallel} and \texttt{netcdf3}
(default) modes, PISM can be built with PnetCDF for best I/O performance. The
option \texttt{-o_format pnetcdf} turns ``on'' PnetCDF I/O code. (PnetCDF seems
to be somewhat fragile, though, so use at your own risk.)


\subsection{Saving time series of scalar diagnostic quantities}
\index{time-series}\index{PISM!saving time-series}
\label{sec:saving-time-series}
\optsection{Saving scalar time-series}

 It is also possible to save time-series of certain scalar diagnostic quantities using a combination of the options \texttt{-ts_file}, \texttt{-ts_times}, and \texttt{-ts_vars}.  For example,
\begin{verbatim}
$ pismr -i foo.nc -y 1e4 -o output.nc -ts_file time-series.nc \
        -ts_times 0:1:1e4 -ts_vars ivol,iareag
\end{verbatim} %$ just to make emacs happy
will run for 10000 years, saving total ice volume and grounded ice area to \texttt{time-series.nc} \emph{yearly}. See tables \ref{tab:time-series-opts} for the list of options and \ref{tab:time-series-1}~and~\ref{tab:time-series-2} for the full list of supported time-series.

Note that, similarly to the snapshot-saving code (section \ref{sec:snapshots}), this mechanism does not affect adaptive time-stepping.  Here, however, PISM will save exactly the number of time-series records requested, \emph{linearly interpolated onto requested times}.

Omitting the \texttt{-ts_vars} option makes PISM save \emph{all} available
variables, as listed in tables
\ref{tab:time-series-1}~and~\ref{tab:time-series-2}.  Because scalar
time-series take minimal storage space, compared to spatially-varying data,
this is usually a reasonable choice. Run PISM with the
\intextoption{list_diagnostics} option to see the list of all available time-series.

If the file \texttt{foo.nc}, specified by \texttt{-ts_file foo.nc}, already exists then by default the existing file will be moved to \texttt{foo.nc~} and the new time series will go into \texttt{foo.nc}.  To append the time series onto the end of the existing file, use option \texttt{-ts_append}.

PISM buffers time-series data and writes it at the end of the run, once 10000
values are stored, or when and \texttt{-extra_file} is saved, whichever comes first. Sending an \texttt{USR1} (or
\texttt{USR2}) signal to a PISM process flushes these buffers, making it
possible to monitor the run. (See section \ref{subsect:signal} for more about
PISM's signal handling.)

\begin{table}[ht]
 \centering
 \begin{tabular}{p{0.35\linewidth}p{0.55\linewidth}}\toprule
    \textbf{Option} & \textbf{Description} \\
    \midrule
    \fileopt{ts_file} & Specifies the file to save to.\\
    \timeopt{ts_times} & Specifies times to save at as a MATLAB-style range $a:\Delta t:b$, a comma-separated list, or a keyword (\texttt{daily}, \texttt{monthly}, \texttt{yearly}). See section \ref{sec:saving-spat-vari}. \\
    \listopt{ts_vars} & Comma-separated list of variables, see
    tables~\ref{tab:time-series-1}~and~\ref{tab:time-series-2}. Omitting this
    option is equivalent to listing the \emph{all} variables.\\
    \intextoption{ts_append} & Append time series to file if it already exists.  No effect if file does not yet exist. \\
    \bottomrule
  \end{tabular}
\caption{Command-line options controlling saving scalar time-series}
\label{tab:time-series-opts}
\end{table}

Besides the above information on usage, here are comments on the physical significance of several scalar diagnostics which appear in tables \ref{tab:time-series-1}~and~\ref{tab:time-series-2}:\index{PISM!physical meaning of scalar diagnostics}
\begin{itemize}
  \item For each variable named \dots\texttt{_flux}, positive values mean ice sheet mass gain.
  \item Ice volume and area are computed and then split among floating and grounded portions: \texttt{ivol} $\mapsto$ (\texttt{ivolf}, \texttt{ivolg}) while \texttt{iarea} $\mapsto$ (\texttt{iareaf},\texttt{iareag}).  The volumes have units \textsl{$m^3$} and the areas have units \textsl{$m^2$}.
  \item The thermodynamic state of the ice sheet can be assessed, in part, by the amount of cold or temperate (``\texttt{temp}'') ice.  Thus their is another splitting: \texttt{ivol} $\mapsto$ (\texttt{ivolcold}, \texttt{ivoltemp}) and \texttt{iarea} $\mapsto$ (\texttt{iareacold},\texttt{iareatemp}).
  \item If a PISM input file contains the \texttt{mapping} variable which has the
\texttt{proj4} attribute containing a PROJ.4 string defining the projection then PISM computes corrected cell areas
using this information, grid parameters, and the WGS84 reference ellipsoid. This compute areas and volumes more accurately.
  \item The sea-level-relevant ice volume \texttt{slvol} is the total grounded ice volume minus the amount of ice, that, in liquid form, would fill up the regions with bedrock below sea level, if this ice were removed.  That is, \texttt{slvol} is the sea level rise potential of the ice sheet at that time.  The result is reported  in sea-level equivalent, i.e.~meters of sea level rise.
  \item Fields \texttt{max_diffusivity} and \texttt{max_hor_vel} relate to PISM time-stepping, and they appear in per-time-step form in the standard output from PISM (i.e.~at default verbosity).  \texttt{max_diffusivity} determines the length of the mass continuity substeps for the SIA stress balance (sub-)model.  \texttt{max_hor_vel} determines the CFL-type restriction for mass continuity and conservation of energy contributions of the SSA stress balance (i.e.~sliding) velocity.
\end{itemize}

\begin{table}[ht]
  \centering
 \begin{tabular}{p{0.4\linewidth}p{0.1\linewidth}p{0.4\linewidth}}
    \toprule
    \textbf{Name} & \textbf{Units} & \textbf{Description}\\
    \midrule
    \texttt{cumulative_grounded_basal_ice_flux} & \textsl{kg} &  cumulative total grounded basal ice flux \\
    \texttt{cumulative_discharge_flux} & \textsl{kg} &  cumulative discharge (calving etc.) flux \\
    \texttt{cumulative_float_kill_flux} & \textsl{kg} &  cumulative \texttt{-float_kill} flux \\
    \texttt{cumulative_nonneg_rule_flux} & \textsl{kg} &  cumulative \emph{numerical} ice flux resulting from enforcing the $\mathrm{thk} \ge 0$ rule \\
    \texttt{cumulative_ocean_kill_flux} & \textsl{kg} &  cumulative \texttt{-ocean_kill} flux \\
    \texttt{cumulative_sub_shelf_ice_flux} & \textsl{kg} &  cumulative total sub-shelf ice flux \\
    \texttt{cumulative_surface_ice_flux} & \textsl{kg} &  cumulative total over ice domain of top surface ice mass flux \\
    \texttt{dimassdt} & \textsl{kg  / s} &  total ice mass rate of change \\
    \texttt{discharge_flux} & \textsl{kg  / s} &  discharge (calving and icebergs) flux \\
    \texttt{divoldt} & \textsl{$m^{3}$  / s} &  total ice volume rate of change \\
    \texttt{dt} & \textsl{second} &  mass continuity time step \\
    \texttt{float_kill_flux} & \textsl{kg  / s} &  \texttt{-float_kill} flux \\
    \texttt{grounded_basal_ice_flux} & \textsl{kg  / s} &  total, over grounded ice, of basal ice flux \\
    \texttt{iarea} & \textsl{$m^{2}$} &  total ice area \\
    \texttt{iareacold} & \textsl{$m^{2}$} &  ice-covered area where basal ice is cold \\
    \texttt{iareaf} & \textsl{$m^{2}$} &  total floating ice area \\
    \texttt{iareag} & \textsl{$m^{2}$} &  total grounded ice area \\
    \texttt{iareatemp} & \textsl{$m^{2}$} &  ice-covered area where basal ice is temperate \\
    \texttt{ienthalpy} & \textsl{J} &  total ice enthalpy \\
    \multicolumn{3}{c}{Continued in Table \ref{tab:time-series-2}}\\
    \bottomrule
  \end{tabular}
\caption{Scalar time-series supported by PISM, part 1}
\label{tab:time-series-1}
\end{table}

\begin{table}[ht]
  \centering
 \begin{tabular}{p{0.4\linewidth}p{0.1\linewidth}p{0.4\linewidth}}
    \toprule
    \textbf{Name} & \textbf{Units} & \textbf{Description}\\
    \midrule
    \multicolumn{3}{c}{Continued from Table \ref{tab:time-series-1}}\\
    \texttt{imass} & \textsl{kg} &  total ice mass \\
    \texttt{ivol} & \textsl{$m^{3}$} &  total ice volume \\
    \texttt{ivolcold} & \textsl{$m^{3}$} &  total volume of cold ice \\
    \texttt{ivolf} & \textsl{$m^{3}$} &  total floating ice volume \\
    \texttt{ivolg} & \textsl{$m^{3}$} &  total grounded ice volume \\
    \texttt{ivoltemp} & \textsl{$m^{3}$} &  total volume of temperate ice \\
    \texttt{max_diffusivity} & \textsl{$m^{2}$  / s} &  maximum diffusivity \\
    \texttt{max_hor_vel} & \textsl{m  / s} &  maximum (absolute) component of horizontal ice velocity over the grid \\
    \texttt{nonneg_rule_flux} & \textsl{kg  / s} &  \emph{numerical} ice flux resulting from enforcing the $\mathrm{thk} \ge 0$ rule \\
    \texttt{ocean_kill_flux} & \textsl{kg  / s} &  \texttt{-ocean_kill} flux \\
    \texttt{slvol} & \textsl{m} &  total sea-level relevant ice \textbf{in sea-level equivalent} \\
    \texttt{sub_shelf_ice_flux} & \textsl{kg  / s} &  total sub-shelf ice flux \\
    \texttt{surface_ice_flux} & \textsl{kg  / s} &  total over ice domain of top surface ice mass flux \\
    \bottomrule
  \end{tabular}
\caption{Scalar time-series supported by PISM, part 2}
\label{tab:time-series-2}
\end{table}


\subsection{Saving time series of spatially-varying diagnostic quantities}
\label{sec:saving-spat-vari}
\optsection{Saving 2D and 3D time-series}
\index{PISM!saving diagnostic quantities regularly}
\index{PISM!diagnostic quantities}

Sometimes it is useful to have PISM save a handful of diagnostic quantities every 10 years (or even every year).  One can use snapshots (section \ref{sec:snapshots}), but doing so can easily fill your hard-drive because snapshots are complete (re-startable) model states.  Sometimes you want a \emph{subset} of model variables saved reasonably-frequently in an output file.

Use options \texttt{-extra_file}, \texttt{-extra_times}, and \texttt{-extra_vars} for this.  For example,
\begin{verbatim}
$ pismr -i foo.nc -y 10000 -o output.nc -extra_file extras.nc \
        -extra_times 0:10:1e4 -extra_vars csurf,cbase
\end{verbatim} %$
will run for 10000 years, saving the magnitude of horizontal velocities at the
ice surface and at the base of ice every 10 years. Times are specified using a
comma-separated list or a MATLAB-style range. Please see table \ref{tab:extras}
for all the options corresponding to this feature;
tables~\ref{tab:three-d-diagnostics},~\ref{tab:two-d-diagnostics-1},~\ref{tab:two-d-diagnostics-2},~and~\ref{tab:two-d-diagnostics-3}
list all the variable choices.

The list of available diagnostic quantities depends on the model setup. For
example, a run with only one vertical grid level in the bedrock thermal layer
will not be able to save \texttt{litho_temp}, an SIA-only run does not use a
basal yield stress model and so will not provide \texttt{tauc}, etc. To see
which quantities are available in a particular setup, use the
\intextoption{list_diagnostics} option, which prints the list of diagnostics
and stops.

In addition to specifying a constant step in \texttt{-extra_times a:step:b} one can save every day, month, or every year by using \texttt{daily}, \texttt{monthly} or \texttt{yearly} instead of a number; for example
\begin{verbatim}
$ pismr -i foo.nc -y 100 -o output.nc -extra_file extras.nc \
        -extra_times 0:monthly:100 -extra_vars dHdt
\end{verbatim} %$
will save the rate of change of the ice thickness every month for 100 years.
PISM uses the Gregorian calendar to compute lengths of months and years. In
this case the \emph{reference date} mentioned in section \ref{sec:i-format}
\textbf{is} used. Use the \intextoption{reference_date} option or the
configuration parameter with the same name to set it. (And see \emph{PISM's
  climate forcing components} for more details.)

It is frequently desirable to save diagnostic quantities at regular intervals for the whole duration of the run; options \intextoption{extra_times}, \intextoption{ts_times}, and \intextoption{save_times} provide a shortcut. For example, use \texttt{-extra_times yearly} to save at the end of every year.

This is especially useful when using a climate forcing file to set run duration:
\begin{verbatim}
$ pismr -i foo.nc -surface given -surface_given_file climate.nc \
        -calendar gregorian -time_file climate.nc \
        -extra_times monthly -extra_file ex.nc -extra_vars thk
\end{verbatim} %$
will save ice thickness at the end of every month while running PISM for the duration of climate forcing data in \texttt{climate.nc}.

Times given using \texttt{-extra_times} are interpreted as endpoints of reporting intervals. This implies that \texttt{0:1:10} will produce 10 records and \emph{not} 11.

If the file \texttt{foo.nc}, specified by \texttt{-extra_file foo.nc}, already exists then by default the existing file will be moved to \texttt{foo.nc\~} and the new time series will go into \texttt{foo.nc}.  To append the time series onto the end of the existing file, use option \texttt{-extra_append}.

Note that this mechanism modifies PISM's adaptive time-stepping scheme to save
\emph{exactly} at the times requested (instead of using linear interpolation).

\begin{table}[ht]
 \centering
 \begin{tabular}{p{0.35\linewidth}p{0.55\linewidth}}\toprule
    \textbf{Option} & \textbf{Description}\\
    \midrule
    \fileopt{extra_file} & Specifies the file to save to; should be different from the output \texttt{o} file.\\
    \timeopt{extra_times} & Specifies times to save at either as a MATLAB-style range $a:\Delta t:b$ or a comma-separated list.\\
    \listopt{extra_vars} & Comma-separated list of variables, see
    tables~\ref{tab:three-d-diagnostics},~\ref{tab:two-d-diagnostics-1},~\ref{tab:two-d-diagnostics-2},~and~\ref{tab:two-d-diagnostics-3} \\
    \intextoption{extra_split} & Save to separate files, similar to \texttt{-save_split}\\
    \intextoption{extra_append} & Append variables to file if it already exists.  No effect if file does not yet exist, and no effect if \intextoption{extra_split} is set. \\
    \bottomrule
  \end{tabular}
\caption{Command-line options controlling extra diagnostic output}
\label{tab:extras}
\end{table}

\begin{table}[ht]
  \centering
  \begin{tabular}{p{0.15\linewidth}p{0.15\linewidth}p{0.6\linewidth}}
    \toprule
    \textbf{Name} & \textbf{Units} & \textbf{Description} \\
    \midrule
    \texttt{age} & \textsl{s} & age of ice \\
    \texttt{enthalpy} & \textsl{J $\mathrm{kg}^{-1}$} & ice enthalpy (includes sensible heat, latent heat, pressure) \\
    \texttt{temp} & \textsl{K} & ice temperature \\
    \texttt{cts} & \textsl{none} &  $\mathrm{cts} = E/E_s(p)$, so cold-temperate transition surface is at $\mathrm{cts} = 1$ \\
    \texttt{liqfrac} & \textsl{1} &  liquid water fraction in ice (between $0$ and $1$) \\
    \texttt{litho_temp} & \textsl{K} & lithosphere (bedrock) temperature, in \texttt{PISMBedThermalUnit} \\
    \texttt{temp} & \textsl{K} &  ice temperature \\
    \texttt{temp_pa} & \textsl{degrees Celsius} &  pressure-adjusted ice temperature (degrees above pressure-melting point) \\
    \texttt{uvel} & \textsl{m / year} &  horizontal velocity of ice in the X direction \\
    \texttt{vvel} & \textsl{m / year} &  horizontal velocity of ice in the Y direction \\
    \texttt{wvel} & \textsl{m / year} &  vertical velocity of ice, relative to geoid \\
    \texttt{wvel_rel} & \textsl{m / year} &  vertical velocity of ice, relative to base of ice directly below \\
    \bottomrule
  \end{tabular}
\caption{3D diagnostic quantities}
\label{tab:three-d-diagnostics}
\end{table}

\begin{table}[ht]
  \centering
  \begin{tabular}{p{0.15\linewidth}p{0.15\linewidth}p{0.6\linewidth}}
    \toprule
    \textbf{Name} & \textbf{Units} & \textbf{Description} \\
    \midrule
    \texttt{climatic_mass_balance} & \textsl{m / year} & ice-equivalent surface mass balance (accumulation/ablation) rate \\
    \texttt{ice_surface_temp} & \textsl{K} & annual average ice surface temperature, below firn processes \\
    \texttt{bedtoptemp} & \textsl{K} & temperature of top of bedrock thermal layer \\
    \texttt{bfrict} & \textsl{W  / $m^2$} &  basal frictional heating \\
    \texttt{bheatflx} & \textsl{W  / $m^2$} & upward geothermal flux at bedrock surface \\
    \texttt{bmelt} & \textsl{m / year} & ice basal melt rate in ice thickness per time \\
    \texttt{bwat} & \textsl{m} & effective thickness of subglacial melt water \\
    \texttt{bwp} & \textsl{Pa} & subglacial (pore) water pressure \\
    \texttt{cbar} & \textsl{m / year} &  magnitude of vertically-integrated horizontal velocity of ice \\
    \texttt{cbase} & \textsl{m / year} &  magnitude of horizontal velocity of ice at base of ice \\
    \texttt{cell_area} & \textsl{$m^{2}$} & cell areas \\
    \texttt{cflx} & \textsl{$m^{2}$ / year} &  magnitude of vertically-integrated horizontal flux of ice \\
    \texttt{csurf} & \textsl{m / year} &  magnitude of horizontal velocity of ice at ice surface \\
    \texttt{dHdt} & \textsl{m / year} &  ice thickness rate of change \\
    \texttt{dbdt} & \textsl{m / year} & bedrock uplift rate \\
    \texttt{diffusivity} & \textsl{$m^{2}$  / s} &  diffusivity of SIA mass continuity equation \\
    \texttt{enthalpybase} & \textsl{J  / kg} &  ice enthalpy at the base of ice \\
    \texttt{enthalpysurf} & \textsl{J  / kg} &  ice enthalpy at 1m below the ice surface \\
    \texttt{h_x} & \textsl{none} &  the x-component of the surface gradient, on the staggered grid\\
    \texttt{h_y} & \textsl{none} &  the y-component of the surface gradient, on the staggered grid\\
    \texttt{hardav} & $Pa\, s^{1/n}$ &  vertical average of ice hardness \\
    \texttt{lat} & \textsl{degrees north} & latitude \\
    \texttt{lon} & \textsl{degrees east} & longitude \\
    \texttt{mask} & \textsl{none} & grounded/dragging/floating integer mask \\
    \texttt{proc_ice_area} & \textsl{none} &  number of cells containing ice in a processor's domain \\
    \texttt{rank} & \textsl{none} &  processor rank \\
   \multicolumn{3}{c}{Continued in Table \ref{tab:two-d-diagnostics-2}}\\
  \bottomrule
  \end{tabular}
  \caption{2D diagnostic quantities, part 1}
  \label{tab:two-d-diagnostics-1}
\end{table}

\begin{table}[ht]
  \centering
  \begin{tabular}{p{0.15\linewidth}p{0.15\linewidth}p{0.6\linewidth}}
    \toprule
    \textbf{Name} & \textbf{Units} & \textbf{Description} \\
    \midrule
    \multicolumn{3}{c}{Continued from Table \ref{tab:two-d-diagnostics-1}}\\
    \texttt{schoofs_theta} & \textsl{1} &  multiplier $\theta$ in \cite{Schoofbasaltopg2003} \\
    \texttt{shelfbmassflux} & \textsl{m / year} & ice mass flux from ice shelf base (positive flux is loss from ice shelf) \\
    \texttt{shelfbtemp} & \textsl{Kelvin} & absolute temperature at ice shelf base \\
    \texttt{strain_rates} & \textsl{1/s} & eigenvalues of the horizontal, vertically-integrated strain rate tensor \\
    \texttt{tauc} & \textsl{Pa} & yield stress for basal till (plastic or pseudo-plastic model) \\
    \texttt{taud} & \textsl{Pa} & driving shear stress at the base of ice, X and Y components \\
    \texttt{taud_mag} & \textsl{Pa} &  magnitude of the driving shear stress at the base of ice \\
    \texttt{tempbase} & \textsl{K} &  ice temperature at the base of ice \\
    \texttt{tempicethk_basal} & \textsl{m} &  thickness of the basal layer of temperate ice \\
    \texttt{tempicethk} & \textsl{m} &  temperate ice thickness (total column content) \\
    \texttt{temppabase} & \textsl{Celsius} &  pressure-adjusted ice temperature at the base of ice \\
    \texttt{tempsurf} & \textsl{K} &  ice temperature at 1m below the ice surface \\
    \texttt{thksmooth} & \textsl{m} &  thickness relative to smoothed bed elevation in \cite{Schoofbasaltopg2003} \\
    \texttt{thk} & \textsl{m} & land ice thickness \\
    \texttt{tillphi} & \textsl{degrees} & friction angle for till under grounded ice sheet \\
    \texttt{topgsmooth} & \textsl{m} &  smoothed bed elevation in \cite{Schoofbasaltopg2003}\\
    \texttt{topg} & \textsl{m} & bedrock surface elevation \\
    \texttt{usurf} & \textsl{m} & ice upper surface elevation \\
    \texttt{velbar} & \textsl{m / year} &  vertical mean of horizontal ice velocity in the X and Y directions \\
    \texttt{velbase} & \textsl{m / year} &  horizontal velocity of ice at the base of ice in the X and Y directions\\
    \texttt{velsurf} & \textsl{m / year} &  horizontal velocity of ice at ice surface in the X and Y directions\\
    \texttt{wvelbase} & \textsl{m / year} &  vertical velocity of ice at the base of ice, relative to the geoid \\
    \texttt{wvelsurf} & \textsl{m / year} &  vertical velocity of ice at ice surface, relative to the geoid \\
   \multicolumn{3}{c}{Continued in Table \ref{tab:two-d-diagnostics-3}}\\
  \bottomrule
  \end{tabular}
  \caption{2D diagnostic quantities, part 2}
  \label{tab:two-d-diagnostics-2}
\end{table}

\begin{table}[ht]
  \centering
  \begin{tabular}{p{0.15\linewidth}p{0.15\linewidth}p{0.6\linewidth}}
    \toprule
    \textbf{Name} & \textbf{Units} & \textbf{Description} \\
    \midrule
    \multicolumn{3}{c}{Continued from Table \ref{tab:two-d-diagnostics-2}}\\
    \texttt{ocean_kill_flux} & \textsl{kg / s} & calving flux due to the ``\texttt{-ocean_kill}'' mechanism \\
    \texttt{ocean_kill_flux_cumulative} & \textsl{Gt} & cumulative version of \texttt{ocean_kill_flux} \\
    \texttt{nonneg_flux_cumulative} & \textsl{Gt} & cumulative numerical flux created by enforcing non-negativity of ice thickness \\
    \texttt{grounded_basal_flux_cumulative} & \textsl{Gt} & cumulative basal flux into the ice in grounded areas \\
    \texttt{floating_basal_flux_cumulative} & \textsl{Gt} & cumulative basal flux into the ice in floating areas \\
  \bottomrule
  \end{tabular}
  \caption{2D diagnostic quantities, part 3}
  \label{tab:two-d-diagnostics-3}
\end{table}


\subsection{Saving re-startable snapshots of the model state}\index{snapshots of the model state}\index{PISM!saving snapshots of the model state}
\label{sec:snapshots}
\optsection{Saving snapshots}
Sometimes you want to check the model state every 1000 years, for example.  One possible solution is to run PISM for a thousand years, have it save all the fields at the end of the run, then restart and run for another thousand, and etc.  This forces the adaptive time-stepping mechanism to stop \emph{exactly} at multiples of 1000 years, which may be desirable in some cases.

If saving exactly at specified times is not critical, then use the \texttt{-save_file} and \texttt{-save_times} options.  For example,
\begin{verbatim}
$ pismr -i foo.nc -y 10000 -o output.nc -save_file snapshots.nc \
        -save_times 1000:1000:10000
\end{verbatim}
starts a PISM evolution run, initializing from \texttt{foo.nc}, running for
10000 years and saving snapshots to \texttt{snapshots.nc} at the first time-step
after each of the years 1000, 2000, \dots, 10000.

We use a MATLAB-style range specification, $a:\Delta t:b$, where $a,\Delta t,b$ are in years.  The time-stepping scheme is not affected, but as a consequence we do not guarantee producing the exact number of snapshots requested if the requested save times have spacing comparable to the model time-steps.  This is not a problem in the typical case in which snapshot spacing is much greater than the length of a typical timestep.

It is also possible to save snapshots at intervals that are not equally-spaced
by giving the \texttt{-save_times} option a comma-separated list. For example,
\begin{verbatim}
$ pismr -i foo.nc -y 10000 -o output.nc -save_file snapshots.nc \
        -save_times 1000,1500,2000,5000
\end{verbatim}
will save snapshots on the first time-step after years 1000, 1500, 2000 and 5000.
The comma-separated list given to the \texttt{-save_times} option can be at most 200 numbers long.

If \texttt{snapshots.nc} was created by the command above, running
\begin{verbatim}
$ pismr -i snapshots.nc -y 1000 -o output_2.nc
\end{verbatim}
will initialize using the last record in the file, at about $5000$ years.  By contrast, to restart from $1500$ years (for example) it is necessary to extract the corresponding record using \texttt{ncks}\index{NCO (NetCDF Operators)!ncks}
\begin{verbatim}
$ ncks -d t,1500years snapshots.nc foo.nc
\end{verbatim}
and then restart from \texttt{foo.nc}.  Note that \texttt{-d t,N} means ``extract the $N$-th record'' (counting from zero).  So, this command is equivalent to
\begin{verbatim}
$ ncks -d t,1 snapshots.nc foo.nc
\end{verbatim}
Also note that the second snapshot will probably be \emph{around} $1500$ years and \texttt{ncks} handles this correctly: it takes the record closest to $1500$ years.

By default re-startable snapshots contain only the variables needed for
restarting PISM. Use the command-line option \texttt{-save_size} to change what is saved.

Another possible use of snapshots is for restarting runs on a batch system which kills jobs which go over their allotted time.  Running PISM with options \texttt{-y 1500} \texttt{-save_times 1000:100:1400} would mean that if the job is killed before completing the whole 1500 year run, we can restart from near the last multiple of $100$ years.  Restarting with option \texttt{-ye} would finish the run on the desired year.

When running PISM on such a batch system it is also possible to save
re-startable snapshots at equal wall-clock time (as opposed to model time)
intervals by adding the ``\txtopt{backup_interval}{hours}'' option.

\textbf{A note of caution:} if the wall-clock limit is equal to $N$ times backup
interval for a whole number $N$ PISM will likely get killed while writing the
last backup.

It is also possible to save snapshots to separate files using the
\texttt{-save_split} option.  For example, the run above can be changed to
\begin{verbatim}
$ pismr -i foo.nc -y 10000 -o output.nc -save_file snapshots \
        -save_times 1000,1500,2000,5000 -save_split
\end{verbatim}
for this purpose.  This will produce files called
\texttt{snapshots-}year\texttt{.nc}.  This option is generally faster if many
snapshots are needed, apparently because of the time necessary to reopen a
large file at each snapshot when \texttt{-save_split} is not used.  Note
that tools like NCO\index{NCO (NetCDF Operators)!wildcards} and
\texttt{ncview}\index{NetCDF!tools!work with wildcards} usually behave as desired with wildcards like ``\texttt{snapshots-*.nc}''.

Table \ref{tab:snapshot-opts} lists the options related to saving snapshots of the model state.

\begin{table}[ht]
  \centering
 \begin{tabular}{p{0.35\linewidth}p{0.55\linewidth}}\toprule
    \textbf{Option} & \textbf{Description} \\
    \midrule
    \fileopt{save_file} & Specifies the file to save to.\\
    \timeopt{save_times} & Specifies times at which to save snapshots, by either a MATLAB-style range $a:\Delta t:b$ or a comma-separated list. \\
    \intextoption{save_split} & Separate the snapshot output into files
    named \texttt{snapshots-}year\texttt{.nc}.  Faster if you are saving more
    than a dozen or so snapshots. \\
    \txtopt{save_size}{[none,small,medium,big]} & similar to \texttt{o_size},
    changes the ``size'' of the file (or files) written; the default is ``small''\\
    \bottomrule
  \end{tabular}
\caption{Command-line options controlling saving snapshots of the model state.}
\label{tab:snapshot-opts}
\end{table}


\subsection{Run-time diagnostic viewers}
\label{sec:diagnostic-viewers}
\optsection{Run-time diagnostic viewers}
Basic graphical views of the changing state of a PISM ice model are available at the command line by using options listed in table \ref{tab:diag-viewers}.  All the quantities listed in tables~~\ref{tab:three-d-diagnostics},~\ref{tab:two-d-diagnostics-1},~\ref{tab:two-d-diagnostics-2},~and~\ref{tab:two-d-diagnostics-3} are available.  Additionally, a couple of diagnostic quantities are \emph{only} available as run-time viewers; these are shown in table \ref{tab:special-diag-viewers}.

Note that (for performance and implementation reasons) map viewers
are transposed.

\begin{table}[ht]
 \centering
  \begin{tabular}{p{0.4\linewidth}p{0.5\linewidth}}\toprule
    \small
    \textbf{Option} & \textbf{Description}\\
    \midrule
    \listopt{view_map} & Turns on map-plane views of one or several variables, see tables~\ref{tab:three-d-diagnostics},~\ref{tab:two-d-diagnostics-1},~\ref{tab:two-d-diagnostics-2},~and~\ref{tab:two-d-diagnostics-3}  \\
    \txtopt{view_size}{number} & desired viewer size, in pixels\\
    \listopt{view_sounding} &Turns on sounding viewers showing values in a column\\
    \txtopt{id}{row} & Sounding row-index\\
    \txtopt{jd}{column} & Sounding column-index\\
    \intextoption{display} & The option \texttt{-display :0} seems to
    frequently be needed to let PETSc use Xwindows when running multiple
    processes.  \emph{It must be given as a \emph{final} option, after all the
      others.}\\
   \bottomrule
    \normalsize
  \end{tabular}
\caption{Options controlling run-time diagnostic viewers}
\label{tab:diag-viewers}
\end{table}
The option \texttt{-view_map} shows map-plane views of 2D fields and surface
and basal views of 3D fields (see tables~\ref{tab:three-d-diagnostics},~\ref{tab:two-d-diagnostics-1},~\ref{tab:two-d-diagnostics-2},~and~\ref{tab:two-d-diagnostics-3}); for example:
\begin{verbatim}
$ pismr -i input.nc -y 1000 -o output.nc -view_map thk,tempsurf
\end{verbatim}
shows ice thickness and ice temperature at the surface.

The option \texttt{-view_sounding} allows viewing a \emph{sounding} of a variable at a given grid point:
\begin{verbatim}
$ pismr -i input.nc -y 1000 -o output.nc -view_sounding temp
\end{verbatim}
shows the dependence of the ice temperature on $z$ (more precisely, on the
number of the vertical grid layer) at the center of the domain. One can use options \texttt{-id} and \texttt{-jd} to choose a different point.

Related to soundings, one can expose all the linear systems being solved in ice columns by using \texttt{-view_sys} along with the sounding location (\texttt{-id XX} and \texttt{-jd YY}).  For example, if the enthalpy formulation is in use then a \Matlab-format text file \texttt{enth_iXX_jYY.m} will be generated at each time step.  % FIXME:  add year to file name?
This text file can be run as a script in \Matlab.  The linear system will be stored in a matrix, a right-hand-side vector, and the solution in a solution vector.  This way these linear systems can be checked for debugging purposes.

\begin{table}[ht]
  \centering
 \begin{tabular}{p{0.35\linewidth}p{0.55\linewidth}}\toprule
    \small
    \textbf{Variable name or an option} & \textbf{Description}\\\midrule
  \intextoption{ssa_view_nuh} & log base ten of \texttt{nuH}, only available
    if the finite-difference SSA solver is active. You can adjust the viewer
    size with \txtopt{ssa_nuh_viewer_size}{\emph{number}}. \\
    \intextoption{ksp_monitor_draw} & Iteration monitor for the Krylov subspace routines (KSP) in PETSc. Residual norm versus iteration number.\\\bottomrule
    \normalsize
  \end{tabular}
\caption{Special run-time-only diagnostic viewers}
\label{tab:special-diag-viewers}
\end{table}


\subsection{PISM's default flags and parameters (and how to change them)}
\label{sec:pism-defaults}
\optsection{PISM configuration file}

PISM's behavior depends on values of many flags and physical parameters (see
\href{http://www.pism-docs.org/doxy/html/index.html}{PISM Source Code Browser} for details). Most of parameters have default values\footnote{For \texttt{pismr}, grid parameters $Mx$, $My$, $Mz$, $Mbz$, $Lz$, $Lbz$, that have to be set at bootstrapping, are some exceptions.} which are read from the configuration file \texttt{pism_config.nc} in the \texttt{lib} sub-directory.

It is possible to run PISM with an alternate configuration file using the \fileopt{config} command-line option:
\begin{verbatim}
$ pismr -i foo.nc -y 1000 -config my_config.nc
\end{verbatim}

The file \texttt{my_config.nc} has to contain \emph{all the flags and parameters} present in \texttt{pism_config.nc}.

The list of parameters is too long to include here; please see the \href{http://www.pism-docs.org/doxy/html/index.html}{PISM Source Code Browser} for an automatically-generated table describing them.

Some command-line options \emph{set} configuration parameters; some PISM
executables have special parameter defaults. To see the configuration used in a
PISM run, use the option \fileopt{dump_config}. The resulting file will have values
used, no matter \emph{how} they were set.

\subsection{Managing parameter studies}
\label{sec:parameter-studies}
Keeping all PISM output files in a parameter study straight can be a challenge.

If parameters of interest can be controlled using command-line options, one can use \texttt{ncdump -h} and look at the \texttt{history} global attribute.

Changing parameters not available at the command-line is also possible~--- by using a custom configuration file (see section \ref{sec:pism-defaults}). The problem is that this makes it hard to see which parameters were changed.

The \fileopt{config_override} command-line option provides an alternative. First of all, a file used with this option can have \emph{a subset} of flags and parameters present in \texttt{pism_config.nc}. Moreover, PISM adds the \texttt{pism_overrides} variable present in this file to the output file, making it easy to see which parameters were used to produce it.

Here's an example: suppose we want to compare the dynamics of an ice-sheet on Earth to the same ice-sheet on Mars.\footnote{This is an admittedly simplified example. One will surely need to change more than just the acceleration due to gravity for a real comparison.}

Running
\begin{verbatim}
$ pismr -i input.nc -y 1e5 -o earth.nc <other PISM options>
\end{verbatim}
produces the ``Earth'' result, since PISM's defaults correspond to this planet.

Next, we create \texttt{mars.cdl}, containing the following:
\small
\begin{verbatim}
netcdf mars {
    variables:
    byte pism_overrides;
    pism_overrides:standard_gravity = 3.728;
    pism_overrides:standard_gravity_doc = "m s-2; standard gravity on Mars";
}
\end{verbatim}
\normalsize
Notice that the variable name is \texttt{pism_overrides} and not \texttt{pism_config} above. Now
\begin{verbatim}
$ ncgen -o mars_config.nc mars.cdl
$ pismr -i input.nc -y 1e5 -config_override mars_config.nc \
        -o mars.nc <other PISM options>
\end{verbatim}
will create \texttt{mars.nc}, the result of the ``Mars'' run.

In this case, we can use \texttt{ncdump} to see what was different about \texttt{mars.nc}:
\small
\begin{verbatim}
$ ncdump -h mars.nc |grep pism_overrides
 byte pism_overrides ;
    pism_overrides:standard_gravity_doc = "m s-2; standard gravity on Mars" ;
    pism_overrides:standard_gravity = 3.728 ;
macbook:pism>
\end{verbatim}
\normalsize

\subsection{Regridding}
\label{sec:regridding}
\optsection{Regridding}
\optseealso{Bootstrapping}

It is common to want to interpolate a coarse grid model state onto a finer grid or vice versa.  For example, one might want to do the EISMINT II experiment on the default grid, producing output \texttt{foo.nc}, but then interpolate both the ice thickness and the temperature onto a finer grid.  The basic idea of ``regridding'' in PISM is that one starts over from the beginning on the finer grid, but one extracts the desired variables stored in the coarse grid file and interpolates these onto the finer grid before proceeding with the actual computation.

The transfer from grid to grid is reasonably general---one can go from coarse to fine or vice versa in each dimension $x,y,z$---but the transfer must always be done by \emph{interpolation} and never \emph{extrapolation}.  (An attempt to do the latter will always produce a PISM error.)

Such ``regridding'' is done using the \fileopt{regrid_file} and
\listopt{regrid_vars} commands as in this example: \index{executables!\texttt{pisms}}

\begin{verbatim}
$  pisms -eisII A -Mx 101 -My 101 -Mz 201 -y 1000 \
         -regrid_file foo.nc -regrid_vars thk,temp -o bar.nc
\end{verbatim}
\noindent By specifying regridded variables ``\texttt{thk,temp}'', the ice thickness and temperature values from the old grid are interpolated onto the new grid.  Here one doesn't need to regrid the bed elevation, which is set identically zero as part of the EISMINT II experiment A description, nor the ice surface elevation, which is computed as the bed elevation plus the ice thickness at each time step anyway.

A slightly different use of regridding occurs when ``bootstrapping'', as described in section \ref{sec:boot} and illustrated by example in section \ref{sec:start}.

See table \ref{tab:regridvar} for the regriddable variables using
\texttt{-regrid_file}.  Only model state variables are regriddable, while climate and boundary data generally are not explicitly regriddable.  (Bootstrapping, however, allows the same general interpolation as this explicit regrid.)

\begin{table}[ht]
  \centering
  \begin{tabular}{ll}\toprule
    \textbf{Name} & \textbf{Description}\\ \midrule
    \texttt{age} & age of ice\\
    \texttt{bwat} & effective thickness of subglacial melt water \\
    \texttt{bmelt} & basal melt rate \\
    \texttt{dbdt} & bedrock uplift rate \\
    \texttt{litho_temp} & lithosphere (bedrock) temperature \\
    \texttt{mask} & grounded/dragging/floating integer mask, see section \ref{subsect:floatmask} \\
    \texttt{temp} & ice temperature \\
    \texttt{thk} & land ice thickness \\
    \texttt{topg} & bedrock surface elevation \\
    \texttt{enthalpy} & ice enthalpy\\
    \bottomrule
    \normalsize
  \end{tabular}
\caption{Regriddable variables.\index{regrid}  Use \texttt{-regrid_vars} with these names.}
\label{tab:regridvar}
\end{table}

Here is another example: suppose you have an output of a PISM run on a fairly
coarse grid (stored in \texttt{foo.nc}) and you want to continue this run on a
finer grid. This can be done using \texttt{-regrid_file} along with
\texttt{-boot_file}\index{refining the grid}:
\begin{verbatim}
$ pismr -boot_file foo.nc -Mx 201 -My 201 -Mz 21 -Lz 4000 \
        -regrid_file foo.nc -regrid_vars litho_temp,enthalpy -y 100 -o bar.nc \
        -surface constant
\end{verbatim}
In this case all the model-state 2D variables present in \texttt{foo.nc} will
be interpolated onto the new grid during bootstrapping, which happens first,
while three-dimensional variables are filled using heuristics mentioned in
section \ref{sec:boot}.  Then temperature in bedrock (\texttt{litho_temp}) and
ice enthalpy (\texttt{enthalpy}) will be interpolated from \texttt{foo.nc} onto the
new grid during the regridding stage, overriding values set at the
bootstrapping stage.  All of this, bootstrapping and regridding, occurs before
the first timestep.


\newcommand\pid{\textsl{pid}s}

\subsection{Signals, to control a running PISM model} \label{subsect:signal} \index{signals} \index{PISM!catches signals -TERM,-USR1,-USR2}  Ice sheet model runs sometimes take a long time, so the state of a run may need checking.  Sometimes the run needs to be stopped, but with the possibility of restarting.  PISM implements these behaviors using ``signals'' from the POSIX standard, included in Linux and most flavors of Unix.  Table \ref{tab:signals} summarizes how PISM responds to signals.  A convenient form of \texttt{kill}, for Linux users, is \texttt{pkill} which will find processes by executable name.  Thus ``\texttt{pkill -USR1 pismr}'' might be used to send all PISM processes the same signal, avoiding an explicit list of \pid.

\begin{table}[ht]
\centering
\begin{tabular}{llp{0.60\linewidth}}\toprule
\textbf{Command} & \textbf{Signal} & \textbf{PISM behavior} \\
\midrule
\texttt{kill -KILL} \pid & \texttt{SIGKILL} & Terminate with extreme prejudice. PISM cannot catch it and no state is saved. \\
\texttt{kill -TERM} \pid & \texttt{SIGTERM} & End processes, but save the last model state in the output file, using \texttt{-o} name or default name as normal.  Note that the \texttt{history} string in the output file will contain an ``\texttt{EARLY EXIT caused by signal SIGTERM}'' indication. \\
\texttt{kill -USR1} \pid & \texttt{SIGUSR1} & Allow process(es) to continue, but save the model state at the current time as ``\texttt{pism}\textsl{X}\texttt{-}\textsl{year}\texttt{.nc}''.  Time-stepping is not altered.  Also flushes time-series output buffers. \\
\texttt{kill -USR2} \pid & \texttt{SIGUSR2} & Just flush time-series output buffers.\index{signals!USR2} \\
\bottomrule
\end{tabular}
\caption{Signalling running PISM processes.  ``\pid''~stands for list of all identifiers of the PISM processes.}
\label{tab:signals}
\end{table}

Here is an example.  Suppose we start a long verification run in the
background, with standard out redirected into a file: \index{executables!\texttt{pismv}}

\begin{verbatim}
pismv -test G -Mz 101 -y 1e6 -o testGmillion.nc >> log.txt &
\end{verbatim}

\noindent This run gets a Unix process id,\index{signals!pid (process id)}, which we assume is ``8920''.  (Get it using \texttt{ps} or \texttt{pgrep}.)  If we want to observe the run without stopping it we send the \texttt{USR1} signal:\index{signals!USR1}

\begin{verbatim}
kill -USR1 8920
\end{verbatim}

\noindent Suppose it happens that we caught the run at year 31871.5.  Then, for example, a NetCDF file \texttt{pismv-31871.495.nc} is produced.  Note also that in the standard out log file \texttt{log.txt} the line

\begin{verbatim}
caught signal SIGUSR1:  Writing intermediate file ... and flushing time series.
\end{verbatim}
\noindent appears around that time step.  Suppose, on the other hand, that the run needs to be stopped.  Then a graceful way is\index{signals!term}

\begin{verbatim}
kill -TERM 8920
\end{verbatim}

\noindent because the model state is saved and can be inspected.



\subsection{Understanding adaptive time-stepping} \label{subsect:adapt}
\index{PISM!adaptive time stepping scheme}
\optsection{Adaptive time-stepping}

At each time step the PISM standard output includes ``flags'' and then a summary of the model state using a few numbers.  A typical example is
\small
\begin{verbatim}
 y SIA v$Eh d (dt=1.07878 in 2 substeps; av dt_sub=0.53939)
  SSA:     4 outer iterations
S  -3456.41429:  3.35908  2.1664   2748.786  271.2114
\end{verbatim}
\normalsize
\noindent The format of the third line, the summary, is simple:
\small
\begin{verbatim}
S         YEAR:     ivol     iarea     max_diff
\end{verbatim}
\normalsize
That is, we have the total ice volume, total ice area, and the maximum diffusivity of the SIA mass conservation.  A ``\texttt{U}'' line near the beginning of the standard output indicates the units.

The characters ``\texttt{ y SIA v\$Eh}'' at the beginning of the flag line give a very terse description of which physical processes were modeled in that time step.  Both the SIA and SSA stress balances were solved, and the latter, which is the most computationally-expensive, describes its nonlinear iterations.

Now we explain what the rest of the flags line, namely

``\texttt{d (dt=1.07878 in 2 substeps; av dt_sub=0.53939)}''

\noindent might mean.  Note that the PISM time step is determined by an
adaptive mechanism.  PISM does each step explicitly when
numerically-approximating mass conservation in the map-plane.  This, and the
modeling of horizontal advection in conservation of energy and age evolution,
requires such an adaptive time-stepping scheme \cite{BBL}.  The first character
we see here, namely ``\texttt{d}'', is the adaptive-timestepping ``reason''
flag.  See Table \ref{tab:adaptiveflag}; ``\texttt{d}'' means that the time
step was limited by the diffusivity of SIA mass conservation.  We also see that
there was a major time step of 1.08 model years divided into 2 substeps of
about 0.54 years.  The \intextoption{skip} option enables this mechanism, while
\intextoption{skip_max} sets the maximum number of such substeps. The adaptive
mechanism may choose to take fewer substeps to satisfy numerical stability
criteria, however.

\begin{table}[ht]
\centering
\begin{tabular}{p{0.05\linewidth}p{0.9\linewidth}}\toprule
\textbf{Flag} & \textbf{Active adaptive constraint} \\ \midrule
\texttt{c} & three-dimensional CFL for temperature/age advection \cite{BBL} \\
\texttt{d} & diffusivity for SIA mass conservation \cite{BBL,HindmarshPayne} \\
\texttt{e} & end of prescribed run time \\
\texttt{f} & \texttt{-dt_force} set; generally option \texttt{-dt_force}, which overrides the adaptive scheme; \emph{should not be used}  \\
\texttt{m} & maximum allowed $\Delta t$ applies; set with \texttt{-max_dt} \\
\texttt{t} & maximum $\Delta t$ was temporarily set by a derived class, or by the mechanism which saves time-series of spatially-varying quantities \\
\texttt{u} & 2D CFL for mass conservation in SSA regions (upwinded; \cite{BBssasliding})\\
\bottomrule
\normalsize
\end{tabular}
\caption{Meaning of the adaptive time-stepping ``reason'' flag in the standard output flag line.}
\label{tab:adaptiveflag}
\end{table}

\begin{table}[ht]
  \centering
 \begin{tabular}{lp{0.7\linewidth}}
    \toprule
    \textbf{Option} & \textbf{Description} \\
    \midrule
    \intextoption{adapt_ratio} & Adaptive time stepping ratio for the explicit
    scheme for the mass balance equation. \\
    \txtopt{max_dt}{(years)} & The maximum time-step in years.  The adaptive
    time-stepping scheme will make the time-step shorter than this as needed
    for stability, but not longer.\\
    \txtopt{dt_force}{(years)} & The time step (in years) to take, overriding the
    adaptive scheme. \emph{Not recommended.}\\
    \intextoption{skip} & Enables time-step skipping, see below. \\
    \intextoption{skip_max} & Number of mass-balance steps, including SIA
    diffusivity updates, to perform before temperature, age, and SSA
    stress balance computations are done.  This is only effective if the time
    step is being limited by the diffusivity time step restriction associated
    to mass continuity using the SIA.  The maximum recommended value for
    \texttt{-skip_max} is, unfortunately, dependent on the context.  The
    temperature field should be updated when the surface changes significantly,
    and likewise the basal sliding velocity if it comes (as it should) from the
    SSA calculation.\\
    \bottomrule
  \end{tabular}
\caption{Options controlling time-stepping}
\label{tab:time-stepping}
\end{table}

\subsection{PETSC options for PISM users}\label{subsect:petscoptions}
\optsection{PETSC options for PISM users}

All PETSc programs including PISM accept command line options which control how PETSc distributes jobs among parallel processors, how it solves linear systems, what additional information it provides, and so on.  The PETSc manual \cite{petsc-user-ref} is the complete reference on these options.  We list some here that are useful to PISM users.  They can be mixed in any order with PISM options.

Both for PISM and PETSc options, there are ways of avoiding the inconvenience of long commands with many runtime options.  Obviously, and as illustrated by examples in the previous sections, shell scripts can be set up to run PISM.  But PETSc also provides two mechanisms to give runtime options without retyping at each run command.

First, the environment variable \texttt{PETSC_OPTIONS} can be set.  For example, a sequence of runs might need the same refined grid, and you might want to know if other options are read, ignored, or misspelled.  Set (in bash):

\texttt{export PETSC_OPTIONS="-Mx 101 -My 101 -Mz 51 -options_left"}

\noindent The runs 
\begin{verbatim}
$ pismv -test F -y 100
$ pismv -test G -y 100
\end{verbatim}
\noindent then have the same refined grid in each run, and the runs report on which options were read.

Alternatively, the file \texttt{.petscrc} is always read, if present, from the directory where PISM (i.e.~the PETSc program) is started.  It can have a list of options, one per line.   In theory, these two PETSc mechanisms (\verb|PETSC_OPTIONS| and \verb|.petscrc|) can be used together.

% "-da_processors_x M -da_processors_y N" should not be documented in this sub-appendix
% because they do not work.  the reason is that IceModelVec2 and IceModelVec3 put 
% the Mx, My dimensions in different arguments to the DACreate commands

Now we address controls on how PETSc solves systems of linear equations, which uses the PETSc ``KSP'' component (Krylov methods).  Such linear solves are needed each time the nonlinear SSA stress balance equations are used (e.g.~with options \texttt{-ssa_sliding} and \texttt{-ssa_floating_only}).

Especially for solving the SSA equations with high resolution on multiple processors, it is recommended that the option \intextoption{ksp_rtol} be set lower than its default value of $10^{-5}$.  For example, 

\begin{verbatim}
$  mpiexec -n 8 ssa_testi -Mx 3 -My 769 -ssa_method fd
\end{verbatim}

\noindent fails to converge on a certain machine, but adding ``\verb|-ksp_rtol 1e-10|'' works fine.

There is also the question of solver \emph{type}, using option \intextoption{ksp_type}.  Based on one processor evidence from \texttt{ssa_testi}, the following are possible choices in the sense that they work and allow convergence at some reasonable rate: \texttt{cg}, \texttt{bicg}, \texttt{gmres}, \texttt{bcgs}, \texttt{cgs}, \texttt{tfqmr}, \texttt{tcqmr}, and \texttt{cr}.  It appears \texttt{bicg}, \texttt{gmres}, \texttt{bcgs}, and \texttt{tfqmr}, at least, are all among the best.  The default is \texttt{gmres}.

Actually the KSP uses preconditioning.  This aspect of the solve is critical for parallel scalability, but it gives results which are dependent on the number of processors.  The preconditioner type can be chosen with \intextoption{pc_type}. Several choices are possible, but for solving the ice stream and shelf equations we recommend only \texttt{bjacobi}, \texttt{ilu}, and \texttt{asm}.  Of these it is not currently clear which is fastest; they are all about the same for \texttt{ssa_testi} with high tolerances (e.g.~\texttt{-ssa_rtol 1e-7} \texttt{-ksp_rtol 1e-12}).  The default (as set by PISM) is \texttt{bjacobi}.  To force no preconditioning, which removes processor-number-dependence of results but may make the solves fail, use \texttt{-pc_type none}.


\subsection{Utility and test scripts} \label{subsect:scripts}\index{python scripts} In the \verb|test/| and \verb|util/| subdirectories of the PISM directory the user will find some python scripts and one Matlab script, listed in Table \ref{tab:scripts-overview}.  The python scripts are all documented at the \textsl{Packages} tab on the \href{http://www.pism-docs.org/doxy/html/index.html}{PISM Source Code Browser}.  The python scripts all take option \texttt{--help}.

\newcommand{\scripthead}[1]{\texttt{#1}}

\begin{table}[ht]
  \centering
 \begin{tabular}{p{0.35\linewidth}p{0.65\linewidth}}
    \toprule
    \textbf{Script} & \textbf{Function(s)}\\
    \midrule
    \scripthead{test/vfnow.py} & Organizes the process of verifying PISM.  Specifies standard refinement paths for each of the tests (section \ref{sec:verif}). \\
    \scripthead{test/vnreport.py} & Automates the creation of convergence graphs like figures \ref{fig:thickerrsB}--~\ref{fig:velerrsI}. \\
    \scripthead{util/fill_missing.py} & Uses an approximation to Laplace's equation $\grad^2 u = 0$ to smoothly replace missing values in a two-dimensional NetCDF variable.  The ``hole'' is filled with an average of the boundary non-missing values. Depends on \texttt{netcdf4-python} and \texttt{scipy} Python packages. \\
   \scripthead{util/check_stationarity.py} & \\
    \scripthead{util/nc2mat.py} & Reads specified variables from a NetCDF file and writes them to an output file in the MATLAB binary data file format \texttt{.mat}, supported by MATLAB version 5 and later.  Depends on \texttt{netcdf4-python} and \texttt{scipy} Python packages. \\
    \scripthead{util/nccmp.py} & a script comparing variables in a given pair
    of NetCDF files; used by PISM software tests\\
    \scripthead{util/pism_config_editor.py} & a script that makes modifying or
    creating PISM configuration files easier \\
    \scripthead{util/pism_matlab.m} & an example MATLAB script showing how to
    create a simple NetCDF file PISM can bootstrap from\index{bootstrapping!preparing data using MATLAB}\\
   \bottomrule
  \end{tabular}
\caption{Some scripts which help in using PISM.}
\label{tab:scripts-overview}
\end{table}

\subsection{Using PISM for flow-line modeling}
\label{sec:flowline-modeling}
\optsection{Flow-line modeling}

As described in sections \ref{subsect:coords} and \ref{subsect:grid}, PISM is a
three-dimensional model. Moreover, parameters
\texttt{Mx} and \texttt{My} have to be greater than or equal to three, so it is
not possible to turn PISM into a 2D (flow-line) model by setting \texttt{Mx} or
\texttt{My} to 1.

There is a way around this, though: by using the \intextoption{periodicity}
option to tell PISM to make the computational grid $y$-periodic and providing
initial and boundary conditions that are functions of $x$ only one can ensure
that there is no flow in the $y$-direction. (Option \intextoption{periodicity}
takes an argument specifying the direction: \texttt{none}, \texttt{x},
\texttt{y} and \texttt{xy}--- for ``periodic in both X- and Y-directions''.)

In this case \texttt{Mx} can be any number; we want to avoid unnecessary
computations, though, so ``\texttt{-Mx 3}'' is the obvious choice.

One remaining problem is that PISM still expects input files to contain both
\texttt{x} and \texttt{y} dimensions. To help with this, PISM comes with a
Python script \texttt{flowline.py} that turns NetCDF files with $N$ grid points
along a flow line into files with 2D fields containing $N\times3$ grid
points.\footnote{This script requires \texttt{numpy}, \texttt{netCDF3} or
  \texttt{netCDF4} Python modules. Please run \texttt{flowline.py --help} for a
  full list of options.}

Here's an example. The script below creates a minimal (and obviously
unphysical) dataset. It describes ice geometry (variables \texttt{thk} and
\texttt{topg}) and the boundary conditions required by the \texttt{constant}
surface model (variables \texttt{climatic_mass_balance} and
\texttt{ice_surface_temp}; see section \emph{PISM's climate forcing components}). \footnote{We recommend creating more metadata-rich
  datasets; this script contains the bare minimum to make it shorter.}
\begin{verbatim}
#!/usr/bin/env python
from numpy import linspace, minimum, maximum, abs
import netCDF4

Lx         = 1000e3                     # 1000 km
Mx         = 501
topg_slope = -1e-4
thk_0      = 1e3                        # meters
climatic_mass_balance_0 = 0             # m / year
ice_surface_temp_0      = -10           # Celsius

nc = netCDF4.Dataset("slab.nc", 'w')
nc.createDimension('x', Mx)

x    = nc.createVariable('x',    'f4', ('x',))
topg = nc.createVariable('topg', 'f4', ('x',))
thk  = nc.createVariable('thk',  'f4', ('x',))
climatic_mass_balance = nc.createVariable('climatic_mass_balance', 'f4', ('x',))
ice_surface_temp = nc.createVariable('ice_surface_temp', 'f4', ('x',))

x[:]    = linspace(-Lx, Lx, Mx);    x.units = "m"
topg[:] = topg_slope * (x[:] - Lx); topg.units = "m"

thk[:] =  maximum(minimum(5e3 - abs(x[:])*0.01, thk_0),0)
thk.units = "m"

climatic_mass_balance[:] = climatic_mass_balance_0;
climatic_mass_balance.units = "m / year"
ice_surface_temp[:] = ice_surface_temp_0;
ice_surface_temp.units = "Celsius"

nc.close()
\end{verbatim}

The file \texttt{slab.nc}, created by this script, is not ready to use with
PISM. However, running
\begin{verbatim}
$ flowline.py -o slab-in.nc --expand -d y slab.nc
\end{verbatim} %$
produces  a PISM-ready \texttt{slab-in.nc}--- by telling \texttt{flowline.py} to
``expand'' its input file in the ``y''-direction.

Now we can ``bootstrap'' from \texttt{slab-in.nc}:
\begin{verbatim}
$ mpiexec -n 2 pismr -surface constant -boot_file slab-in.nc \
    -Mx 201 -My 3 -Lx 1000 -Ly 4 -Lz 2000 -Mz 11 -periodicity y \
    -y 10000 -o pism-out.nc
\end{verbatim}%$

To make it easier to visualize data in the file created by PISM, ``collapse'' it:
\begin{verbatim}
$ flowline.py -o slab-out.nc --collapse -d y pism-out.nc
\end{verbatim}%$

\subsection{Managing source code modifications}
\label{sec:code-modifications}

``Practical usage'' may include editing the source code to extend, fix
or replace parts of PISM.

We provide both user-level (this manual) and developer-level documentation.
Please see source code browsers at \url{http://www.pism-docs.org} for the latter.

\begin{itemize}
\item To use your (modified) version of PISM, you will need to follow the
  compilation from sources instructions in the \emph{Installation Manual}
\item We find it very useful to be able to check if a recent source code change
  broke something. PISM comes with ``regression tests'', which check if certain
  parts of PISM perform the way it should.\footnote{This automates running
    verification tests described in section \ref{sec:verif}, for example.}

  Run ``\texttt{make test}'' in the build directory to run PISM's regression tests.

  Note, though, that while a test failure usually means that the new code needs
  more work, passing all the tests does not guarantee that everything works as
  it should. We are constantly adding new tests, but so far only a subset
  of PISM's functionality can be tested automatically.
\item We strongly recommend using a version control system to manage code
  changes. Not only is it safer than the alternative, it is also more efficient.
\end{itemize}

%%% Local Variables: 
%%% mode: latex
%%% TeX-master: "manual"
%%% End: 

% LocalWords:  NetCDF Regridding emacs ncview pclimate startable
