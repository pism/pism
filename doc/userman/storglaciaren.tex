
\section{Example: Modeling Storglaci{\"a}ren, a mountain glacier}\label{sec:storglaciaren} \index{Storglaci{\"a}ren}
\optsection{Storglaci{\"a}ren}

Storglaci{\"a}ren\footnote{``Storglaci{\"a}ren'' means ``big glacier'' in Swedish.} is a small valley glacier in northern Sweden (Figure~\ref{fig:storglaciaren}).  It is approximately 3.2\,km long and 1\,km wide, smaller than a single grid cell in most Greenland models.  Most of the glacier is temperate, except for a cold near-surface layer in the ablation zone.  Such a thermal structure is often-called a Scandinavian-type structure.  Thanks to the nearby Tarfala research station it is one of the best investigated glaciers worldwide, and the wealth of data available makes it suitable for modeling studies.  Here we demonstrate how PISM can be used for valley glaciers in a flow-line mode, and we show that PISM's conservation of energy scheme is able to simulate the glacier's thermal structure.

\begin{figure}[ht]
  \centering
  \includegraphics[width=3.in,keepaspectratio=true]{storglaciaren}\qquad
  \includegraphics[width=2.75in,keepaspectratio=true]{storglaciaren-dem}
  \caption{Storglaci{\"a}ren, northern Sweden. Left: photo by R. Hock. Right: digital elevation model.}
  \label{fig:storglaciaren}
\end{figure}

To get started, change directories and run the preprocess script:
\begin{quote}\small
\begin{verbatim}
$ cd examples/storglaciaren/
$ ./preprocess.sh
\end{verbatim}
\normalsize\end{quote}
This reads the digital elevation model from ASCII files and generates the necessary input files for both the 3-dimensional and the flow-line application.  The file \texttt{psg_config.cdl} is also transformed to \texttt{psg_config.nc}.  It contains several \texttt{pism_overrides} parameter choices suitable for small valley glaciers.  Specifically, we increase the limit above which drainage occurs from 1\,\% to 2\,\% liquid water fraction.

The just-generated flowline bootstrapping file is \texttt{pism_storglaciaren_flowline.nc}.  The variable \texttt{ice_surface_temp} in this file uses the mean annual air temperature $T_{\mathrm{MA}}=-6^{\circ}$C from the nearby Tarfala Research Stations below the firn line ($z_{\textrm{FL}} = 1400$\,m above sea level), but, above the firnline where the ice is temperate, we use 0$^{\circ}$C.

So let's run the flow-line example on 2 MPI processes:
\begin{quote}\small
\begin{verbatim}
$ ./psg_flowline.sh 2 &> out.stor &
\end{verbatim}
\normalsize\end{quote}

The first two runs smooth the surface and generate a near-steady enthalpy field.

At the next stage we attempt to infer the mass balance that has the present day geometry as a steady-state (for now, we ignore the fact that Storglaci{\"a}ren is probably not in a steady-state).  For this purpose we can use PISM's mass balance modifier:
\begin{quote}\small
\begin{verbatim}
-surface constant,forcing -force_to_thickness_file psg_flowline_35m_steady.nc -force_to_thickness_alpha 0.05
\end{verbatim}
\normalsize\end{quote}
The result of this run is shown in Figure~\ref{fig:storglaciaren-ftt-result}. Without much parameter fine-tuning, simulated surface velocities are reasonably close to observations, c.f. \cite{AschwandenBlatter}.  We also see that a mass balance of about 4 meters per year at the bergschrund, almost linearly decreasing to -4.5 meters per year at the tongue, is required to maintain the glacier's geometry.  However, observations show that a more realistic present day mass balance decreases lineafly from 2.5 meters per year at the bergschrund to -3 meters per year at the tongue.  We now use the surface processes model which is enabled by \texttt{-surface elevation} and allows to define temperature and surface mass balance as a function of surface elevation (see the \emph{PISM's climate forcing components} document for details).  The next run uses this present day mass balance which is imposed via 
\begin{quote}\small
\begin{verbatim}
-acab -3,2.5.,1200,1450,1615 -acab_limits -3,0
\end{verbatim}
\normalsize\end{quote} 
For simplicity we assume that the firnline does not change with time. After 25 years the glacier has thinned in the accumulation area but the signal has not yet reached the ablation zone (Figure~\ref{fig:storglaciaren-25a-result}). (Plots were produced with the script \texttt{plot_flowline_results.py}.)

\begin{figure}[ht]
  \centering
  \includegraphics[width=5.in,keepaspectratio=true]{sg-flowline-ftt-result}
  \caption{Storglaci{\"a}ren, northern Sweden.  Modeled present day.  Upper panel: horizontal surface velocity (blue solid), basal sliding velocity (blue dashed), and modified surface mass balance (purple) along flowline in meters per year. Lower panel: thermal structure. Red colors indicate liquid water content, blue colors are temperature in degrees Celsius.}
  \label{fig:storglaciaren-ftt-result}
\end{figure}

\begin{figure}[ht]
  \centering
  \includegraphics[width=5.in,keepaspectratio=true]{sg-flowline-25a-result}
  \caption{Storglaci{\"a}ren, northern Sweden. 25\,a model run with present-day surface mass balance. Upper panel: horizontal surface velocity (blue solid), basal sliding velocity (blue dashed), and surface mass balance (purple) along flowline in meters per year. Lower panel: thermal structure. Red colors indicate liquid water content, blue colors are temperature in degrees Celsius.}
  \label{fig:storglaciaren-25a-result}
\end{figure}
