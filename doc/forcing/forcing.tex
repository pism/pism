\documentclass[titlepage,letterpaper,final]{scrartcl}

\usepackage{scrindex}           % multiple index support using the "index" package
\usepackage{index}

%% THE FOLLOWING SHOULD CHANGE FOR A STABLE RELEASE:
\newcommand{\PISMREV}{revision \input{revision.tex}}
\newcommand{\PETSCREL}{3.2}
\newcommand{\PISMDOWNLOADMSG}{Get development branch source code: \quad\texttt{git clone -b dev git@github.com:pism/pism.git pism-dev} \quad}
\newcommand{\PISMBROWSERURL}{http://www.pism-docs.org/wiki/doku.php?id=browser}

%\addtolength\topmargin{-.1in}
\addtolength\textheight{0.75in}
\addtolength{\oddsidemargin}{-.4in}
\addtolength{\evensidemargin}{-.4in}
\addtolength{\textwidth}{0.9in}
\newcommand{\normalspacing}{\renewcommand{\baselinestretch}{1.1}\tiny\normalsize}
\newcommand{\tablespacing}{\renewcommand{\baselinestretch}{1.0}\tiny\normalsize}
\normalspacing

\usepackage[usenames]{xcolor}

\usepackage{bm,url,xspace,verbatim}
\usepackage{amssymb,amsmath}
\usepackage[pdftex]{graphicx}

\usepackage{booktabs}           % better rules in tables
\usepackage{xtab}               % long (multi-page) tables

%% uncomment to see locations of index entries
% \proofmodetrue

\usepackage{underscore}

% this lets us avoid the scrartcl/hyperref conflict...
\let\ifvtex\relax

% hyperref should be the last package we load
\usepackage[pdftex,
                colorlinks=true,
                plainpages=false, % only if colorlinks=true
                linkcolor=blue,   % only if colorlinks=true
                citecolor=blue,   % only if colorlinks=true
                urlcolor=blue     % only if colorlinks=true
]{hyperref}

\newcommand{\ddt}[1]{\ensuremath{\frac{\partial #1}{\partial t}}}
\newcommand{\ddx}[1]{\ensuremath{\frac{\partial #1}{\partial x}}}
\newcommand{\ddy}[1]{\ensuremath{\frac{\partial #1}{\partial y}}}
\renewcommand{\t}[1]{\texttt{#1}}
\newcommand{\Matlab}{\textsc{Matlab}\xspace}
\newcommand{\bq}{\mathbf{q}}
\newcommand{\bU}{\mathbf{U}}
\newcommand{\eps}{\epsilon}
\newcommand{\grad}{\nabla}
\newcommand{\Div}{\nabla\cdot}

%% macros having to do with documentation for options; note these appear in the index

\newindex{default}{idx}{ind}{General Index}
\newindex{options}{odx}{ond}{PISM Command-line options}

\def\optsection#1{%
  \def\optindex##1{\index[options]{#1!##1}}
  \def\optseealso##1{\index[options]{#1|see{##1}}}
}

\optsection{FIXME}

% Use this to index option definitions:
\newcommand{\intextoption}[1]{\texttt{-#1}\optindex{\texttt{-#1}}}

\newcommand{\txtopt}[2]{\texttt{-#1} #2\optindex{\texttt{-#1} #2}}

\newcommand{\listopt}[1]{\txtopt{#1}{\emph{comma-separated list}}}
\newcommand{\fileopt}[1]{\txtopt{#1}{\emph{filename}}}
\newcommand{\timeopt}[1]{\txtopt{#1}{\emph{range or list}}}


\pdfinfo{
/Title (PISM Climate Forcing options)
/Author (the PISM authors)
/Subject (Setting up PISM's climate forcing options)
/Keywords (PISM ice sheet modeling climate forcing)
}

\begin{document}

\begin{titlepage}

  \begin{center}
    {\huge\usekomafont{title} PISM's climate forcing options}
    \vspace{0.5cm}

    {\Large The PISM Authors}
    \vspace{1cm}

    \vfill

    \small Support by email: \texttt{help\@@pism-docs.org}. 
    \medskip

    Manual date \today. Based on PISM \PISMREV.
    \medskip

    \PISMDOWNLOADMSG
  \end{center}
\end{titlepage}

\newpage
\phantom{bob}

\begin{center}
  Please see the \emph{PISM User's Manual} for the list of authors
\end{center}

\vspace{0.2in}
\begin{quote}
\textsl{Copyright (C) 2004--2012 The PISM Authors}
\medskip

\noindent \textsl{This file is part of PISM.  PISM is free software; you can redistribute it and/or modify it under the terms of the GNU General Public License as published by the Free Software Foundation; either version 2 of the License, or (at your option) any later version.  PISM is distributed in the hope that it will be useful, but WITHOUT ANY WARRANTY; without even the implied warranty of MERCHANTABILITY or FITNESS FOR A PARTICULAR PURPOSE.  See the GNU General Public License for more details.  You should have received a copy of the GNU General Public License\index{GPL (\emph{GNU Public License})} along with PISM; see \emph{\texttt{COPYING}}.  If not, write to the Free Software Foundation, Inc., 51 Franklin St, Fifth Floor, Boston, MA  02110-1301 USA}
\end{quote}

\newpage
\setcounter{tocdepth}{3}
\small
\tableofcontents
\normalsize

\newpage


\section{Introduction}\label{sect:intro}

\intextoption{blah}
\cite{AIAAGuidelines}


% References and indices
\clearpage\newpage
\bibliography{ice-bib}
\bibliographystyle{siam}

\phantomsection
\addcontentsline{toc}{section}{General Index}
\label{sect:index}
\printindex

\phantomsection
\addcontentsline{toc}{section}{PISM Command-line options}
\printindex[options]

\end{document}
