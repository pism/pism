\documentclass[12pt,final]{amsart}

\addtolength\topmargin{-.1in}
\addtolength\textheight{0.3in}
\addtolength{\oddsidemargin}{-.65in}
\addtolength{\evensidemargin}{-.65in}
\addtolength{\textwidth}{1.3in}
\newcommand{\normalspacing}{\renewcommand{\baselinestretch}{1.1}\tiny\normalsize}
\newcommand{\tablespacing}{\renewcommand{\baselinestretch}{1.0}\tiny\normalsize}
\normalspacing

\usepackage{bm,url,xspace,verbatim}
\usepackage{amssymb,amsmath}
\usepackage[final,pdftex]{graphicx}
\usepackage[pdftex]{hyperref}

\usepackage{natbib}

\newcommand{\ddt}[1]{\ensuremath{\frac{\partial #1}{\partial t}}}
\newcommand{\ddx}[1]{\ensuremath{\frac{\partial #1}{\partial x}}}
\newcommand{\ddy}[1]{\ensuremath{\frac{\partial #1}{\partial y}}}
\renewcommand{\t}[1]{\texttt{#1}}
\newcommand{\Matlab}{\textsc{Matlab}\xspace}
\newcommand{\bU}{\mathbf{U}}
\newcommand{\eps}{\epsilon}
\newcommand{\grad}{\nabla}

\newcommand{\optoptdef}[3]{\vspace{1mm}\noindent \large\texttt{-#1},\,\,\texttt{--#2=}\normalsize\,\,[\textsl{#3}]:\quad}

\newcommand{\rawopt}[1]{\vspace{1mm}\noindent \large\texttt{-#1}\normalsize}
\newcommand{\opt}[1]{\rawopt{#1}\,:\quad}
\newcommand{\optdef}[2]{\rawopt{#1}\,[\textsl{#2}]:\quad}
\newcommand{\optrestrict}[2]{\rawopt{#1}\,[\texttt{#2} \textsl{only}]:\quad}
\newcommand{\optdefrestrict}[3]{\rawopt{#1}\,[\textsl{#2}]\,[\texttt{#3} \textsl{only}]:\quad}
\newcommand{\und}{$\underline{\,\,\,}$}

% note \beginV and \Vend are a pair, but they must be used as follows:
%   \beginV
%      ... stuff
%   \end{verbatim}
%   \Vend
% that is, "\end{verbatim}" still has to appear on a line by itself with no leading spaces
%\newcommand{\Vend}{ \rule{4.6in}{0.1mm}\end{quote} }
%\newcommand{\beginV}{ \begin{quote}\rule{4.6in}{0.1mm}\begin{verbatim} }
\newcommand{\Vend}{ \rule{4.6in}{0.1mm}\end{quote}\normalsize }
%\newcommand{\beginV}{ \small\begin{quote}\rule{4.6in}{0.1mm}\begin{verbatim} }
\newcommand{\beginV}{ \scriptsize\begin{quote}\rule{4.6in}{0.1mm}\begin{verbatim} }

\newcommand{\Vfile}[1]{ \begin{quote}\rule{4.6in}{0.1mm} \verbatiminput{#1} \rule{4.6in}{0.1mm}\end{quote} }

%\makeindex

\title[]{\protect{\large Parallel Ice Sheet Model (PISM):\normalsize} \\ \protect{\large\medskip A Summary\normalsize}}

\begin{document}
\maketitle
\thispagestyle{empty}
%\tablespacing

PISM simulates the dynamic evolution of ice sheets like those in Antarctic and Greenland.

The flow of interior ice sheets, ice streams, and ice shelves can each be approximated.  The model is always shallow but different stress balances are solved for these modes.  The flow is fully thermomechanically coupled.  PISM also computes the age of the ice, earth deformation, stored liquid water in the basal till, and other quantities.   Verification tests for many subsystems, including large coupled subsystems, are built in.

PISM solves the flow equations in parallel; it is the only comprehensive ice sheet model which does so.  PISM uses PETSc (\href{http://www-unix.mcs.anl.gov/petsc/petsc-as/}{www-unix.mcs.anl.gov/petsc/petsc-as}) and MPI for parallel computation.  PISM has been run on 500 processors simultaneously to simulate whole-sheet Antarctic ice sheet flow on a 5 km grid, for example.

PISM has been designed with coupling to global circulation models (GCMs) in mind, though this capability is not yet proven.  The input and output format for PISM is CF 1.0-compliant NetCDF (\href{http://www.unidata.ucar.edu/software/netcdf/}{www.unidata.ucar.edu/software/netcdf}).  Because PISM is already an MPI program, coupling to GCMs can potentially be both offline and in parallel.

\subsection*{The ice flow models in PISM}  Significant features of the continuum model approximated by the PISM include:\begin{itemize}
\item The thermomechanically coupled shallow ice approximation equations \citep{Hutter,EISMINT00} are solved by PISM.  The numerical solution of these equations can be verified using highly nontrivial and coupled exact solutions \citep{BLKCB,BBL}.  These verification tests are built in and can be used at any time.
\item Ice shelves and ice streams are modeled by shallow equations which describe flow constrained by membrane stresses and basal sliding resistance (in streams).  PISM solves the equations which were originally established for ice shelves \citep{Morland}, but it also solves the form adapted to ice streams (``dragging ice shelves'').  Ice streams can be modeled using till described by a range of models from linear viscosity to perfect plasticity \citep{MacAyeal,SchoofStream}.  The solution is verifiable using nontrivial ice stream and ice shelf exact solutions.
\item The regions of grounded ice in which the ice stream model is applied can be determined from a plastic till assumption and the associated free boundary problem \citep{SchoofStream}.
\item A three dimensional age field is computed, a temperature model for the bedrock under an ice sheet is included, and geothermal flux which varies in the map-plane can be used.
\item Within the shallow ice sheet regions the model can use the constitutive relation of Goldsby and Kohlstedt \citep{GoldsbyKohlstedt}.  For inclusion in this flow law, grain size is approximated using a age-grain size relation from the Vostok core \citep{VostokCore}.
\item An advanced bed deformation model is included \citep{BLKfastearth,LingleClark}.  It generalizes the better known elastic lithosphere, relaxing asthenosphere (``ELRA'') model \citep{Greve2001}.  This model can be initialized by an observed bed uplift map, a capability of no other ice sheet model.\end{itemize}

Most of the internal numerical methods are of finite difference type.  The regular and rectangular three-dimensional grid is automatically broken into one-processor portions at run-time; the user merely specifies the number of available processors.  These portions communicate at their boundaries, and such communication is handled by PISM and PETSc, invisibly to the user.

\subsection*{PISM as an ice sheet modeling language}  \begin{itemize}
\item PISM is open source: \url{https://gna.org/projects/pism/}.
\item A complete seventy page \emph{User's Manual} includes EISMINT-Greenland \citep{RitzEISMINT} and EISMINT-Ross \citep{MacAyealetal} modeling intercomparisons as realistic tutorial examples, as well as complete documentation on command line options and on installation.
\item Example scripts for converting real data into PISM input files in NetCDF format are included and documented.
\item Verification tests and simplified geometry experiments (\citep{EISMINT00}, for example) can be run at the command line.
\item The grid can be chosen at the command line.  Regridding can be done at any time, for example taking the result of a rough grid computation and interpolating it onto a finer grid or vice versa.
\item PISM adaptively and automatically determines stable time steps.
\item The state of the model can be viewed graphically at runtime or by examining output files in NetCDF or \Matlab format.
\item The C++ and C source code itself is well-documented.  Comments in the source code comprise a sixty page \emph{Reference Manual} for programmers.  
\item C++ programmers can write derived classes of PISM without having to create ice sheet modeling ``infrastructure'' from scratch.  The open source code of PISM is a natural starting point for new parallel ice sheet models with, for instance, different flow laws, basal physics, or ``higher-order'' stress balances.
\end{itemize}

\small
%         References
\bibliography{ice_bib}
%\bibliographystyle{siam}
\bibliographystyle{igs}
\normalsize

\vfill
\noindent\emph{Learn more about PISM at:}
\bigskip

\centerline{\href{http://www.pism-docs.org/}{www.pism-docs.org}}
\vfill
\scriptsize
\noindent \emph{Summary author:} Ed Bueler, \href{mailto:ffelb@uaf.edu}{\texttt{ffelb@uaf.edu}}.  \hfill \emph{Date:} \today.


\begin{comment}

THIS BLOCK OF TEXT WAS FORMERLY A PART OF THE USER'S MANUAL.  PROBABLY NOT NEEDED BUT KEPT HERE FOR FUTURE REFERENCE.


\clearpage\newpage
\section{Inside PISM: overviews of continuum models and numerical schemes}\label{sect:over}

This section is an executive summary.  Technical documentation of the continuum models, numerical methods, and the source code structure is in the PISM \emph{Reference Manual}.\index{PISM!\emph{Reference Manual}}

\subsection{The continuum models in PISM}  Significant features of the continuum model approximated by the PISM include:\index{PISM!continuum models inside}\begin{itemize}
\item The thermocoupled shallow ice approximation equations \cite{Fowler} is verifiably \cite{BLKCB,BBL} solved.
\item Ice shelves and ice streams are modeled by shallow equations which describe flow by longitudinal strain rates and basal sliding.  These equations are different from the shallow ice approximation.  In shelves and streams the velocities are independent of depth within the ice.  The equations were originally established for ice shelves \cite{Morland,MorlandZainuddin,MacAyealetal,WeisGreveHutter}.  They were adapted for ice streams, as ``dragging ice shelves,'' by MacAyeal \cite{MacAyeal}; see also \cite{HulbeMacAyeal,SchoofStream}.  The solution is verifiable using ice stream and ice shelf exact solutions.
\item The regions of grounded ice in which the ice stream model is applied can be determined from a plastic till assumption and the associated free boundary problem \cite{SchoofStream}.
\item A three dimensional age field is computed.
\item A temperature model for the bedrock under an ice sheet is included and geothermal flux which varies in the map-plane can be used.
\item Within the shallow ice sheet regions the model can use the constitutive relation of Goldsby and Kohlstedt \cite{GoldsbyKohlstedt,Peltieretal}.  For inclusion in this flow law, grain size can be computed using a age-grain size relation from the Vostok core data \cite{VostokCore}, for example.
\item The Lingle-Clark \cite{BLKfastearth,LingleClark} bed deformation model can be used.  It generalizes the better known elastic lithosphere, relaxing asthenosphere (``ELRA'') model \cite{Greve2001}.  This model can be initialized by an observed bed uplift map \cite{BLKfastearth}, or even an uplift map computed by an external model like that in \cite{IvinsJames2005}.\end{itemize}

Many of the parts of the model described above are optional.  For instance, options can choose between five different flow laws; see Appendix \ref{sect:options} for options.

The following features are \emph{not} included in the continuum model, and would (or will) require major additions:
\begin{itemize}
\item Inclusion of all components of the stress tensor through either an intermediate order scheme (e.g.~\cite{Blatter,Hindmarsh06,Pattyn03}) or the full Stokes equations \cite{Fowler}.
\item A model for water--content within the ice.  In the current model the ice is \emph{cold} and not \emph{polythermal}; compare \cite{Greve}.  (On the other hand, in the current model the energy used to melt the ice within a given column, if any, is conserved.  In particular, a layer of basal melt water evolves by conservation of energy in the column.  This layer can activate basal sliding and its latent heat energy is available for refreezing.)
\item A model for basal water mass conservation in the map-plane; compare \cite{JohnsonFastook}.
\item A fully spherical Earth deformation model, for example one descended from the Earth model of \cite{Peltier}.
\end{itemize}


\subsection{The numerical schemes in PISM}  Significant features of the numerical methods in PISM include:\index{PISM!numerical methods inside}\begin{itemize}
\item Verification \cite{Roache} is a primary concern and is built into the code.  Nontrivial verifications are available for isothermal ice sheet flow \cite{BLKCB}, thermocoupled sheet flow \cite{BB,BBL}, conservation in ice and bedrock \cite{BuelerTestK}, the earth deformation model \cite{BLKfastearth}, the coupled (ice flow)/(earth deformation) system in an isothermal and pointwise isostasy  case \cite{BLKfastearth}, ice shelf flow (preprint), and ice stream flow based on a plastic bed \cite{SchoofStream}.
\item The code is \emph{structurally} parallel because the PETSc toolkit is used at all levels \cite{petsc-user-ref}.  PETSc manages the MPI-based communication between processors.  It provides an interface to parallel numerical linear algebra.
\item The grid can be chosen at the command line.  Regridding can be done at any time, for example taking the result of a rough grid computation and interpolating it onto a finer grid or vice versa.
\item A moving boundary technique is used for the temperature equation which does not stretch the vertical in a singular manner; the Jenssen \cite{Jenssen} change of variables is not used.
\item The shallow shelf approximation (SSA) \cite{WeisGreveHutter} is used to determine velocity in the ice shelf and ice stream regions.  Like the full non-Newtonian Stokes equations, they are nonlinear and nonlocal equations for the velocity given the geometry of the streams and shelves and given the ice temperature.  They are solved by straightforward iteration of linearized equations with numerically-determined (vertically-averaged) viscosity.  Either plastic till or linear drag is allowed.  As with all of PISM, the numerical approximation is finite difference.  The linearized finite difference equations are solved by any of the Krylov subspace methods in PETSc \cite{petsc-user-ref}, choosable at the command line.
\item The model uses an explicit time stepping method for flow and a partly implicit method for temperature.  Advection of temperature is upwinded \cite{MortonMayers}.  As described  in  \cite{BBL}, reasonably rigorous stability criteria are applied to the time-stepping scheme, including a diffusivity-based criteria for the explicit mass continuity scheme and a CFL criteria \cite{MortonMayers} for temperature/age advection.  The contributions to mass continuity from sliding of the base, including essentially all of the flow in the ice shelves, is approximated by an upwinding scheme.
\item The local truncation error is generically first order (i.e.~$O(\Delta x,\Delta y,\Delta z,\Delta t)$).
\item The bed deformation model is implemented by a new Fourier collocation (spectral) method \cite{BLKfastearth}.
\item Implementation is in C++ and is object-oriented.  For example, verification occurs in a derived class of the base class.  Also EISMINT II, EISMINT-Greenland, and EISMINT-Ross are each derived classes.
\end{itemize}

\end{comment}

\end{document}

