\documentclass{amsart}
\usepackage{amsmath}
\usepackage{amssymb}
\usepackage{hyperref}
\usepackage{underscore}
\usepackage[svgnames]{xcolor}
\usepackage[margin=1in]{geometry}
\usepackage{fancybox}

% disable indenting the first line of a paragraph
\parindent=0in
\parskip=0.5\baselineskip

% show sub-sub-sections in toc
\setcounter{tocdepth}{4}

% trace of a matrix
\DeclareMathOperator{\Tr}{tr}

% allow align to span page breaks
\allowdisplaybreaks[1]

% vectors
\newcommand{\N}{\mathbf{n}}
\newcommand{\U}{\mathbf{u}}
\newcommand{\T}{\boldsymbol{\tau}}
% area integral
\newcommand{\I}{\ensuremath{\int_{\Omega}}}
% boundary integral
\newcommand{\boundary}{\partial \Omega_N}
\newcommand{\bI}{\ensuremath{\int_{\boundary}}}
% effective stress tensor
\newcommand{\etaM}{\ensuremath{\eta M}}
% partial derivatives
\newcommand{\diff}[2]{\ensuremath{\frac{\partial #1}{\partial #2}}}
% sum over quadrature points
\newcommand{\sumQ}{\ensuremath{\sum_{q = 1}^{N_q}}}
% basis expansion of a partial derivative
\newcommand{\diffbasisexpansion}[3]{\ensuremath{\sum_{#3 = 1}^{N_k} {#1}_{#3} \diff{\phi_{#3}}{#2} }}
\newcommand{\UX}{\diffbasisexpansion{u}{x}{m}}
\newcommand{\UY}{\diffbasisexpansion{u}{y}{m}}
\newcommand{\VX}{\diffbasisexpansion{v}{x}{m}}
\newcommand{\VY}{\diffbasisexpansion{v}{y}{m}}
% drag coefficient as a function of velocity
\newcommand{\betaU}{\beta(\U)}

% basal shear stress
\newcommand{\basalshearstress}[1]{\T_{b#1}}
\newcommand{\taub}{\basalshearstress{}}
\newcommand{\taubx}{\basalshearstress{,x}}
\newcommand{\tauby}{\basalshearstress{,y}}

% basal driving stress
\newcommand{\drivingstress}[1]{\T_{d#1}}
\newcommand{\taud}{\drivingstress{}}
\newcommand{\taudx}{\drivingstress{,x}}
\newcommand{\taudy}{\drivingstress{,y}}

\newcommand{\highlight}[1]{{\color{red!80!black} \fbox{$ \displaystyle #1 $} }}

\newcommand{\R}{\mathbb{R}}

\begin{document}

\title{SSA FEM implementation notes}
\author{Constantine Khroulev}
\maketitle
\tableofcontents


\section{Notation}
\label{sec-1}

\begin{center}
\begin{tabular}{lp{0.5\textwidth}}
Symbol & Meaning\\
\hline
$B$ & vertically-averaged ice hardness\\
$g$ & acceleration due to gravity\\
$H$ & ice thickness\\
$h$ & ice top surface elevation\\
$n$ & Glen flow law exponent\\
$N_k$ & number of trial functions per element\\
$N_q$ & number of quadrature points per element\\
$q$ & sliding power law exponent\\
$w_q$ & weight corresponding to a quadrature point $q$\\
$J_q$ & Jacobian of the map from the reference element to the physical element, evaluated at a quadrature point $q$\\
$\U = (u,v)$ & horizontal ice velocity\\
$\epsilon_{\beta}$, $\epsilon_{\nu}$, $\epsilon_{\eta}$ & regularization parameters for $\betaU$, $\nu$, $\eta$\\
$\nu$ & effective viscosity of ice\\
$\phi$ & trial functions\\
$\psi$ & test functions\\
$\rho$ & ice density\\
$\taub$ & basal shear stress\\
$\taud$ & driving shear stress\\
\end{tabular}
\end{center}

Formulas that appear in the code are \highlight{highlighted.}

\section{The shallow shelf approximation}
\label{sec:ssa-strong}

\newcommand{\Mux}{4u_x + 2v_y}
\newcommand{\Muy}{u_y + v_x}
\newcommand{\Mvx}{u_y + v_x}
\newcommand{\Mvy}{2u_x + 4v_y}
Define the effective SSA strain rate tensor $M$ \cite{Dukowiczetal2010}:
\begin{equation}
  \label{eq:1}
  M =
    \begin{pmatrix}
      \Mux & \Muy \\
      \Mvx & \Mvy \\
    \end{pmatrix}.
\end{equation}
Then the strong form of the SSA system (without boundary conditions) is
\begin{align}
  \label{eq:2}
  - \nabla\cdot(\eta\, M) & = \taub + \taud,\\
  \label{eq:3}
  \eta &= \highlight{ \epsilon_{\eta} + \nu\, H }.
\end{align}
This is equivalent to the more familiar form (found in \cite{SchoofStream}, for example):
\begin{align*}
 - \Big[ (\eta (\Mux))_x + (\eta (\Muy))_y \Big] & = \taubx + \taudx, \\
 - \Big[ (\eta (\Mvx))_x + (\eta (\Mvy))_y \Big] & = \tauby + \taudy.\\
\end{align*}

Here $\taud = \highlight{ \rho g H \nabla h }$ is the gravitational driving shear stress; see subsections for definitions of $\taub$ and the ice viscosity $\nu$.


\subsection{Ice viscosity}
\label{sec:ice-viscosity}

Let $U = \{u,v,w\}$ and $X = \{x, y, z\}$.

The three-dimensional strain rate tensor $D$ (\cite{GreveBlatter2009}, equations 3.25 and 3.29) is defined by

\begin{equation*}
  D_{i,j}(\U) = \frac 12 \left( \diff{U_i}{X_j} + \diff{U_j}{X_i} \right).
\end{equation*}

We assume that ice is incompressible, so $w_z = - (u_x + v_y)$. Moreover, in the shallow shelf approximation horizontal velocity components do not vary with depth, so $u_z = v_z = 0$.

With these assumptions $D$ becomes
\begin{equation*}
  D =
  \begin{pmatrix}
    u_x & \frac{1}{2}\,(u_y + v_x) & 0\\
    \frac{1}{2}\,(u_y + v_x) & v_y & 0\\
    0 & 0 & - (u_x + v_y)\\
  \end{pmatrix}
\end{equation*}

Now the second invariant of $D$ (\cite{GreveBlatter2009}, equation 2.42)
\begin{equation*}
  \gamma = \Tr (D^2) - \left( \Tr D \right)^2
\end{equation*}
simplifies to
\begin{equation}
  \label{eq:4}
  \gamma = \highlight{ \frac{1}{2}\, \left( (u_x)^2 + (v_y)^2 + (u_x + v_y)^2 + \frac{1}{2}\,(u_y + v_x)^2 \right) }.
\end{equation}

We define the regularized effective viscosity of ice $\nu$ (\cite{SchoofStream}, equation 2.3):
\begin{equation}
  \label{eq:5}
  \nu = \highlight{ \frac{1}{2} B \left( \epsilon_{\nu} + \gamma \right)^{(1 - n) / (2n)} },\\
\end{equation}

\subsection{Basal shear stress}
\label{sec:beta}

The basal shear stress is defined by
\begin{equation}
  \label{eq:6}
  \taub =  - \betaU \U,
\end{equation}

where $\beta = \betaU$ is a scalar-valued drag coefficient related to the yield stress.

In PISM, $\betaU$ is defined as follows (see \cite{SchoofHindmarsh}):
\begin{equation}
  \label{eq:7}
  \betaU = \highlight{ \frac{\tau_c}{u_{\text{threshold}}^q}\cdot (\epsilon_{\beta} + |\U|^2)^{(q - 1) / 2} }
\end{equation}


\section{The weak form of the SSA}

Multiplying (\ref{eq:2}) by a test function $\psi$ and integrating by parts, we get the weak form:

\begin{align}
 - \nabla \cdot (\etaM) & = \taud + \taub,\notag \\
 - \I \psi \nabla \cdot (\etaM) & = \I \psi (\taud + \taub),\notag \\
 - \I \nabla \cdot (\psi\, \etaM) + \I \nabla \psi \cdot (\etaM) & = \I \psi (\taud + \taub),\notag \\
 - \bI (\psi\, \etaM)\cdot \N\, ds + \I \nabla \psi \cdot (\etaM) & = \I \psi (\taud + \taub)\notag \\
  \label{eq:8}
  \I \Big[\nabla \psi \cdot (\etaM) - \psi (\taud + \taub)\Big] - \bI (\psi\, \etaM) \cdot \N\, ds &= 0.
\end{align}

Recall that $\psi$ are zero on the Dirichlet part of the boundary, so the boundary integral is equal to the integral over the \emph{Neumann} part of the boundary $\boundary$.

Let $\Phi$ be the line integral above:
\begin{equation}
  \label{eq:9}
  \Phi = \bI (\psi\, \etaM) \cdot \N\, ds
\end{equation}
We keep $\Phi$ in equations below to make these notes easier to digest. See section \ref{sec:boundary} for details of the residual contribution of $\Phi$.

Define $F(u,v)$
\begin{align}
  \label{eq:10}
  F_{1}(u,v) &=  \I \left[ \diff{\psi}{x} \left( \eta (\Mux) \right) + \diff{\psi}{y} \left( \eta (\Muy) \right) - \psi (\taubx + \taudx) \right] - \Phi_{1}\\
  \label{eq:11}
  F_{2}(u,v) &= \I \left[ \diff{\psi}{x} \left( \eta (\Mvx) \right) + \diff{\psi}{y} \left( \eta (\Mvy) \right) - \psi (\tauby + \taudy) \right] - \Phi_{2}.
\end{align}

These notes are concerned with solving the system $F(u,v) = 0$.

\section{Solving the discretized system}
\label{sec-3}

In the following subscripts $x$ and $y$ denote partial derivatives, while subscripts $k$, $l$, $m$ denote nodal values of a particular quantity. Also, to simplify notation from here on $u$, $v$, etc stand for finite element approximations of corresponding continuum variables.

To build a Galerkin approximation of the SSA system, let $\phi$ be trial functions and $\psi$ be test functions\footnote{Note that in a Galerkin approximations $\phi$ and $\psi$ are the same.}. Then we have the following basis expansions:

\begin{equation}
  \label{eq:12}
  \begin{aligned}
    u & = \sum_i \phi_i\, u_i, \\
    v & = \sum_i \phi_i\, v_i,\\
    \diff{u}{x_j} & = \sum_i \diff{\phi_i}{x_j}\ u_i,\\
    \diff{v}{x_j} & = \sum_i \diff{\phi_i}{x_j}\ v_i.\\
  \end{aligned}
\end{equation}

We use Newton's method to solve the system resulting from discretizing equations (\ref{eq:10}) and (\ref{eq:11}). This requires computing residuals and the Jacobian matrix.

\subsection{Residual evaluation}
\label{sec-3-1}

In this and following sections we focus on element contributions to the residual and the Jacobian. Basis functions used here are defined on the reference element, hence the added determinant of the Jacobian of the map from the reference element to a particular physical element $|J_q|$ appearing in all quadratures.

\begin{align}
  \label{eq:13}
  F_{k,1} & = \highlight{ \sumQ |J_q| \cdot w_q\cdot \left[ \eta \left( \diff{\psi_k}{x} (\Mux) + \diff{\psi_k}{y} (\Muy) \right) - \psi_k (\taubx + \taudx) \right]_{\text{evaluated at } q} - \Phi_{k,1} }\\
  \label{eq:14}
  F_{k,2} & = \highlight{ \sumQ |J_q| \cdot w_q\cdot \left[ \eta \left( \diff{\psi_k}{x} (\Mvx) + \diff{\psi_k}{y} (\Mvy) \right) - \psi_k (\tauby + \taudy) \right]_{\text{evaluated at } q} - \Phi_{k,2}}
\end{align}


\subsection{Jacobian evaluation}
\label{sec-3-2}

Equations (\ref{eq:13}) and (\ref{eq:14}) define a map from $\R^{2\times N}$ to $\R^{2\times N}$, where $N$ is the number of nodes in a FEM mesh. To use Newton's method, we need to be able to compute the Jacobian \emph{of this map}.

It is helpful to rewrite equations defining $F_{k,1}$ and $F_{k,2}$ using basis expansions for $u_x$, $u_y$, $v_x$, and $v_y$ (see (\ref{eq:12})), as follows:

\begin{align*}
  F_{k,1} &= \highlight{
            \begin{aligned}[t]
              \sumQ |J_q| \cdot w_q\cdot \Bigg[ &\eta \left(\diff{\psi_k}{x} \left(4 \UX + 2 \VY\right)
                + \diff{\psi_k}{y} \left(\UY + \VX\right)\right) \\
              & - \psi_k \left(\taubx + \taudx\right) \Bigg]_{\text{evaluated at } q} - \Phi_{k,1}\\
            \end{aligned} }\\
  F_{k,2} &= \highlight{
            \begin{aligned}[t]
              \sumQ |J_q| \cdot w_q\cdot \Bigg[ &\eta \left(\diff{\psi_k}{x} \left(\UY + \VX\right)
                + \diff{\psi_k}{y} \left(2 \UX + 4 \VY\right) \right) \\
              & - \psi_k \left(\tauby + \taudy\right) \Bigg]_{\text{evaluated at } q} - \Phi_{k,2}\\
            \end{aligned}}
\end{align*}

The Jacobian has elements
\begin{align*}
  J_{k,l,1} & = \diff{F_{k,1}}{u_l}, & J_{k,l,2} & = \diff{F_{k,1}}{v_l},\\
  J_{k,l,3} & = \diff{F_{k,2}}{u_l}, & J_{k,l,4} & = \diff{F_{k,2}}{v_l}.\\
\end{align*}

\begin{align}
  J_{k,l,1} &= \highlight{
              \begin{aligned}[t]
                \sumQ |J_q| \cdot w_q\cdot \Bigg[&\diff{\eta}{u_l}\cdot
                \left( \diff{\psi_k}{x} (4 u_x + 2 v_y) + \diff{\psi_k}{y} (u_y + v_x) \right)\\
                & + \eta\cdot \left( \diff{\psi_k}{x}\cdot 4 \diff{\phi_l}{x} + \diff{\psi_k}{y}\cdot \diff{\phi_l}{y} \right)
                - \psi_k\cdot \diff{\taubx}{u_l} \Bigg]_{\text{evaluated at } q} - \diff{\Phi_{k,1}}{u_{l}}\\
              \end{aligned} }\\
  J_{k,l,2} &= \highlight{
              \begin{aligned}[t]
                \sumQ |J_q| \cdot w_q\cdot \Bigg[&\diff{\eta}{v_l}\cdot
                \left( \diff{\psi_k}{x} (4 u_x + 2 v_y) + \diff{\psi_k}{y} (u_y + v_x) \right)\\
                & + \eta\cdot \left( \diff{\psi_k}{x}\cdot 2 \diff{\phi_l}{y} + \diff{\psi_k}{y}\cdot \diff{\phi_l}{x} \right)
                - \psi_k\cdot \diff{\taubx}{v_l} \Bigg]_{\text{evaluated at } q} - \diff{\Phi_{k,1}}{v_{l}}\\
              \end{aligned} }\\
  J_{k,l,3} &= \highlight{
              \begin{aligned}[t]
                \sumQ |J_q| \cdot w_q\cdot \Bigg[&\diff{\eta}{u_l}\cdot
                \left( \diff{\psi_k}{x} (u_y + v_x) + \diff{\psi_k}{y} (2 u_x + 4 v_y) \right)\\
                & + \eta\cdot \left( \diff{\psi_k}{x}\cdot \diff{\phi_l}{y} + \diff{\psi_k}{y}\cdot 2 \diff{\phi_l}{x} \right)
                - \psi_k\cdot \diff{\tauby}{u_l} \Bigg]_{\text{evaluated at } q} - \diff{\Phi_{k,2}}{u_{l}}\\
              \end{aligned} }\\
  J_{k,l,4} &= \highlight{
              \begin{aligned}[t]
                \sumQ |J_q| \cdot w_q\cdot \Bigg[&\diff{\eta}{v_l}\cdot
                \left( \diff{\psi_k}{x} (u_y + v_x) + \diff{\psi_k}{y} (2 u_x + 4 v_y) \right)\\
                & + \eta\cdot \left( \diff{\psi_k}{x}\cdot \diff{\phi_l}{x} + \diff{\psi_k}{y}\cdot 4 \diff{\phi_l}{y} \right)
                - \psi_k\cdot \diff{\tauby}{v_l} \Bigg]_{\text{evaluated at } q} - \diff{\Phi_{k,2}}{v_{l}}\\
              \end{aligned} }
\end{align}

In our case the number of trial functions $N_k$ is $4$ ($Q_1$ elements). Our test functions are the same as trial functions (a Galerkin method), i.e. we also have $4$ test functions per element. Moreover, each combination of test and trial functions corresponds to $4$ values in the Jacobian ($2$ equations, $2$ degrees of freedom). Overall, each element contributes to $4 \times 4 \times 4 = 64$ entries in the Jacobian matrix.

To evaluate $J_{\cdot,\cdot,\cdot}$, we need be able to compute the following\footnote{Also $\diff{\Phi_{\cdot,\cdot}}{u}$ and $\diff{\Phi_{\cdot,\cdot}}{v}$; see section \ref{sec:boundary}.}:
\begin{equation*}
  \diff{\eta}{u_l}, \quad \diff{\eta}{v_l}, \quad \diff{\taubx}{u_l},
  \quad \diff{\taubx}{v_l}, \quad \diff{\tauby}{u_l}, \quad \diff{\tauby}{v_l}.
\end{equation*}

Subsections that follow describe related implementation details.


\subsubsection{Ice viscosity}
\label{sec:viscosity-evaluation}

Recall (equation (\ref{eq:3})) that $\eta = \epsilon_{\eta} + \nu H$. We use chain rule to get
\begin{align*}
  \diff{\eta}{u_l} &= \highlight{ H\,\diff{\nu}{\gamma}\cdot \diff{\gamma}{u_l}, } &   \diff{\eta}{v_l} &= \highlight{ H\,\diff{\nu}{\gamma}\cdot \diff{\gamma}{v_l} }.
\end{align*}

The derivative of $\nu$ with respect to $\gamma$ (equation (\ref{eq:4})) can be written in terms of $\nu$ itself:

\begin{align*}
  \diff{\nu}{\gamma} & = \frac{1}{2} B \cdot \frac{1 - n}{2n} \cdot \left(\epsilon_{\nu} + \gamma \right)^{(1 - n) / (2n) - 1}, \\
      & = \frac{1 - n}{2n} \cdot \frac{1}{2} B \left( \epsilon_{\nu} + \gamma \right)^{(1 - n) / (2n)} \cdot \frac{1}{\epsilon_{\nu} + \gamma}, \\
      & = \highlight{ \frac{1 - n}{2n} \cdot \frac{\nu}{\epsilon_{\nu} + \gamma} }.
\end{align*}

To compute $\diff{\gamma}{u_l}$ and $\diff{\gamma}{v_l}$ we need to re-write $\gamma$ (equation (\ref{eq:4})) using the basis expansion (\ref{eq:12}):
\begin{align*}
  \gamma &=
           \begin{aligned}[t]
             \frac{1}{2}\, \Bigg(&\left(\UX\right)^2 + \left(\VY\right)^2 \\
             & + \left(\UX + \VY\right)^2 + \frac{1}{2}\, \left(\UY + \VX\right)^2\Bigg). \\
           \end{aligned}
\end{align*}

So,
\begin{align}
  \diff{\gamma}{u_l} &= u_x \diff{\phi_l}{x} + (u_x + v_y)\diff{\phi_l}{x} + \frac{1}{2} (u_y + v_x)\diff{\phi_l}{y},\notag\\
                     &= \highlight{ (2 u_x + v_y) \diff{\phi_l}{x} + \frac{1}{2} (u_y + v_x)\diff{\phi_l}{y} },\\
  \diff{\gamma}{v_l} &= v_y \diff{\phi_l}{y} + (u_x + v_y)\diff{\phi_l}{y} + \frac{1}{2} (u_y + v_x)\diff{\phi_l}{x},\notag\\
                     &= \highlight{ \frac{1}{2} (u_y + v_x)\diff{\phi_l}{x} + (u_x + 2 v_y)\diff{\phi_l}{y} }.
\end{align}


\subsubsection{Basal drag}
\label{sec:basal-drag}

The method \texttt{IceBasalResistancePlasticLaw::drag_with_derivative()} computes $\beta$ and the derivative of $\beta$ with respect to $\alpha = \frac12 |\U|^2 = \frac12 (u^2 + v^2)$.

So,
\begin{align*}
  \diff{\alpha}{u_l} &= u\cdot \diff{u}{u_l} = u\cdot \phi_l,&
  \diff{\alpha}{v_l} &= v\cdot \diff{v}{v_l} = v\cdot \phi_l.
\end{align*}

Recall from equation (\ref{eq:6})
\begin{align*}
  \taubx &=  \highlight{ - \betaU\cdot u }, &  \tauby &= \highlight{ - \betaU\cdot v }.
\end{align*}

Using product and chain rules, we get
\begin{align*}
  \diff{\taubx}{u_l} &= -\left(\betaU\cdot \diff{u}{u_l} + \diff{\betaU}{u_l}\cdot u\right)\\
                     &= -\left( \betaU\cdot \phi_l + \diff{\betaU}{\alpha}\cdot \diff{\alpha}{u_l}\cdot u \right)\\
                     &= \highlight{ -\left( \betaU\cdot \phi_l + \diff{\betaU}{\alpha}\cdot u^2 \phi_l \right) },\\
  \diff{\taubx}{v_l} &= -\diff{\betaU}{v_l}\cdot u\\
                     &= -\diff{\betaU}{\alpha}\cdot \diff{\alpha}{v_l}\cdot u\\
                     &= \highlight{ -\diff{\betaU}{\alpha}\cdot u\cdot v \cdot \phi_l },\\
  \diff{\tauby}{u_l} &= -\diff{\betaU}{u_l}\cdot v,\\
                     &= \highlight{ -\diff{\betaU}{\alpha}\cdot u\cdot v \cdot \phi_l },\\
  \diff{\tauby}{v_l} &= -\left( \betaU\cdot \diff{v}{v_l} + \diff{\betaU}{u_l}\cdot v \right)\\
                     &= -\left( \betaU\cdot \phi_l + \diff{\betaU}{\alpha}\cdot \diff{\alpha}{v_l}\cdot v \right)\\
                     &= \highlight{ -\left( \betaU\cdot \phi_l + \diff{\betaU}{\alpha}\cdot v^2 \phi_l \right) }.
\end{align*}

\section{Boundary contributions}
\label{sec:boundary}

The boundary contribution consists of the integral
\begin{equation}
  \label{eq:15}
  \Phi = \bI (\psi\, \etaM) \cdot \N\, ds,
\end{equation}
where $M$ is the effective strain rate tensor for the shallow shelf
approximation and $\eta = \epsilon_{\eta} + \nu\, H$.

Define
\newcommand{\dP}{\Delta P}
\begin{equation}
  \label{eq:17}
  \dP = \int_{b}^{h} (p_{\mathrm{ice}} - p_{\mathrm{ocean}})\,dz,
\end{equation}
the ``pressure difference'' at the ice margin. Note that $\dP$ is a function of ice geometry and sea level, but it does \emph{not} depend on the ice velocity and therefore does \emph{not} contribute to the Jacobian.

In our particular application (see \cite{GreveBlatter2009} and \cite{KhroulevBueler2013}), the calving front boundary condition states

\begin{equation}
  \label{eq:16}
  \etaM \cdot \N = \dP\, \N.
\end{equation}

Combining equations (\ref{eq:16}) and (\ref{eq:17}) gives the following expression for $\Phi$:
\begin{equation}
  \label{eq:18}
  \Phi = \bI \psi\, \dP\,ds.
\end{equation}
Note that the right-hand side is a \emph{scalar}, so equation (\ref{eq:18}) gives the expression for \emph{both} $\Phi_{1}$ and $\Phi_{2}$.

\subsection{Residual evaluation}
\label{sec:boundary-residual}

To evaluate the contribution of (\ref{eq:18}) to the residual we need to rewrite this line integral as a regular integral by using a parameterization of $\boundary$.

As usual, we assemble it element-by-element.

\bibliography{ssafem-notes}
\bibliographystyle{siam}

\end{document}
