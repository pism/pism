\documentclass[11pt]{article}
\usepackage{amsmath,hyperref}
\usepackage[margin=1in]{geometry}

\date{}
\title{Calving front stress boundary condition}

\begin{document}

\maketitle
\newcommand{\diff}[2]{\frac{\partial #1}{\partial #2}}
\newcommand{\n}{\mathbf{n}}
\newcommand{\nx}{\n_{x}}
\newcommand{\ny}{\n_{y}}
\newcommand{\nz}{\n_{z}}
\newcommand{\psw}{p_{\text{ocean}}}
\newcommand{\pice}{p_{\text{ice}}}
\newcommand{\rhoi}{\rho_{\text{ice}}}
\newcommand{\rhosw}{\rho_{\text{ocean}}}
\newcommand{\zs}{z_{\text{s}}}
\newcommand{\td}[1]{t^{D}_{#1}}
\newcommand{\D}{\displaystyle}
\newcommand{\dx}{\Delta x}
\newcommand{\dy}{\Delta y}
\newcommand{\viscosity}{\nu}

\section{Notation}
\label{sec-1}

\begin{center}
\begin{tabular}{ll}
\textbf{\textbf{Variable}} & \textbf{\textbf{Meaning}}\\
\hline
$h$ & top surface elevation\\
$b$ & bottom surface elevation\\
$H = h - b$ & ice thickness\\
$\zs$ & sea level elevation\\
$g$ & acceleration due to gravity\\
$\rhoi$ & ice density\\
$\rhosw$ & sea water density\\
$\viscosity$ & vertically-averaged viscosity of ice\\
$\n$ & normal vector\\
$B(T)$ & ice hardness\\
$\pice$ & pressure due to the weight of a column of ice\\
$\psw$ & pressure due to the weight of a column of seawater\\
$D$ & strain rate tensor\\
$d_{e}$ & effective strain rate\\
$t$ & Cauchy stress tensor\\
$t^{D}$ & deviatoric stress tensor\\
\end{tabular}
\end{center}

\section{Calving front stress boundary condition}
\label{sec-2}

In the 3D case the calving front stress boundary condition reads
\begin{equation*}
\left.\left( t \cdot \n \right)\right|_{\text{cf}} = -\psw \n.
\end{equation*}

Expanded in the component form and evaluating the left-hand side at
the calving front and assuming that the calving front face is vertical
($\nz = 0$), this gives
\begin{equation*}
\begin{array}{rcrcl}
(\td{xx} - p) \nx &+& \td{xy} \ny &=& -\psw \nx,\\
\td{xy} \nx &+& (\td{yy} - p) \ny &=& -\psw  \ny,\\
\td{xz} \nx &+& \td{yz} \ny &=& 0.
\end{array}
\end{equation*}

In the hydrostatic approximation shear stresses $\td{xz}$ and $\td{yz}$ are omitted, so we have
\begin{equation*}
\begin{array}{rcrcl}
(\td{xx} - p) \nx &+& \td{xy} \ny &=& -\psw \nx,\\
\td{xy} \nx &+& (\td{yy} - p) \ny &=& -\psw  \ny.\\
\end{array}
\end{equation*}

Next, since $p = p_{\text{column}} - \td{xx} - \td{yy}$,
\begin{equation}
\label{eq:1}
\begin{array}{rcrcl}
(2\td{xx} + \td{yy}) \nx &+& \td{xy} \ny &=& (\pice - \psw) \nx,\\
\td{xy} \nx &+& (2\td{yy} + \td{xx}) \ny &=& (\pice - \psw) \ny.\\
\end{array}
\end{equation}

Now, using the ``viscosity form'' of the flow law
\begin{align*}
\td{} &= 2\viscosity\, D,\\
\viscosity &= \frac 12 B(T) d_{e}^{1/n-1}
\end{align*}
and the fact that in the shallow shelf approximation $u$ and $v$ are
depth-independent, we can define the membrane stress $N$ as the
vertically-integrated deviatoric stress
\begin{align*}
N_{ij} &= \int_{b}^{h} t^{D}_{ij} dz\\
&= 2\, \bar \viscosity\, H\, D_{ij}.\notag
\end{align*}
Here $\bar \viscosity$ is the vertically-averaged effective viscosity.

Integrating \eqref{eq:1} with respect to $z$, we get

\begin{equation}
\label{eq:3}
\begin{array}{rcrcl}
(2N_{xx} + N_{yy}) \nx &+& N_{xy} \ny &=& \displaystyle \int_{b}^{h}(\pice - \psw) dz\, \nx,\\
N_{xy} \nx &+& (2N_{yy} + N_{xx}) \ny &=& \displaystyle \int_{b}^{h}(\pice - \psw) dz\, \ny.
\end{array}
\end{equation}

\section{Shallow shelf approximation}
\label{sec-3}

The shallow shelf approximation written in terms of $N_{ij}$ is

\begin{equation}
\label{eq:2}
\begin{array}{rcrcl}
\D \diff{}{x}(2N_{xx} + N_{yy}) &+& \D \diff{N_{xy}}{y} &=& \D \rho g H \diff{h}{x},\\
\D \diff{N_{xy}}{x} &+& \D \diff{}{y}(2N_{yy} + N_{xx}) &=& \D \rho g H \diff{h}{y}.
\end{array}
\end{equation}

\section{Implementing the CFBC}
\label{sec-4}

We use centered finite difference approximations in the discretization
of the SSA (see \eqref{eq:2}).

Consider the first equation:

\begin{equation}
\label{eq:4}
\diff{}{x}(2N_{xx} + N_{yy}) + \D \diff{N_{xy}}{y} = \D \rho g H \diff{h}{x}.
\end{equation}

\newcommand{\partI}{(2\tilde N_{xx} + \tilde N_{yy})}
\newcommand{\partII}{(\tilde N_{xy})}
Let $\tilde F$ be an approximation of $F$ using a finite-difference
scheme. Then the first SSA equation is approximated by
\begin{equation*}
\frac1{\dx}\left(\partI_{i+\frac12,j} - \partI_{i-\frac12,j}\right) +
\frac1{\dy}\left(\partII_{i,j+\frac12} - \partII_{i,j-\frac12}\right) =
\rho g H \frac{h_{i+\frac12,j} - h_{i-\frac12,j}}{\dx}.
\end{equation*}

Now, assume that the cell boundary (face) at $i+\frac12,j$ is at the
calving front. Then $\n = (1,0)$ and from \eqref{eq:3} we have
\begin{equation*}
2N_{xx} + N_{yy} = \int_{b}^{h}(\pice - \psw) dz.
\end{equation*}
So, instead of assembling a matrix row corresponding to the generic
equation \eqref{eq:4} we use
\begin{equation*}
\frac1{\dx}\left(\int_{b}^{h}(\pice - \psw) dz - \partI_{i-\frac12,j}\right) +
\frac1{\dy}\left(\partII_{i,j+\frac12} - \partII_{i,j-\frac12}\right) =
\rho g H \frac{h_{i+\frac12,j} - h_{i-\frac12,j}}{\dx}.
\end{equation*}

Moving terms that do not depend on the velocity field to the
right-hand side, we get
\begin{equation*}
\frac1{\dx}\left(- \partI_{i-\frac12,j}\right) +
\frac1{\dy}\left(\partII_{i,j+\frac12} - \partII_{i,j-\frac12}\right) =
\rho g H \frac{h_{i+\frac12,j} - h_{i-\frac12,j}}{\dx} + \left.\frac{\int_{b}^{h}(\psw - \pice) dz}{\dx}\right|_{i+\frac12,j}
\end{equation*}

The second equation and other cases ($\n = (-1,0)$, etc) are treated
similarly.

\section{Evaluating the ``pressure imbalance term''}
\label{sec-5}

For $z \in [b, h]$ the column pressures are
\begin{align}
\psw &= 
\begin{cases}
0, & z > \zs,\\
\rhosw\, g (\zs - z), & z \le \zs,
\end{cases}\\
\pice &= \rhoi\, g (h - z).
\end{align}

Depending of the local geometry $b$ is either prescribed (grounded
case) or is a function of the ice thickness and sea level elevation.

\subsection{Floating case}
\label{sec-5-1}

Using the floatation criterion, we have

\begin{equation*}
\int_{b}^{h}{\pice(z)\; dz}-\int_{b}^{\zs}{\psw(z)\; dz}=
 \frac{g\, \rhoi\, \left(\rhosw-\rhoi\right)\, H^2}{2\, \rhosw}
\end{equation*}

\subsection{Grounded below sea level}
\label{sec-5-2}

\begin{equation*}
\int_{b}^{h}{\pice(z)\; dz}-\int_{b}^{\zs}{\psw(z)\; dz}=
 \frac{1}{2}\, g\, \left(\rhoi\, H^2-\rhosw\, \left(\zs-b\right)^2\right)
\end{equation*}

This, in fact, applies in both floating and grounded-below-sea-level
cases.

\subsection{Grounded above sea level}
\label{sec-5-3}

In this case $\psw = 0$, so
\begin{equation*}
\int_{b}^{h}{\pice(z)\; dz}-\int_{b}^{\zs}{\psw(z)\; dz}=\frac{1}{2}\, g\, \rhoi\, H^2
\end{equation*}

\section{Modeling melange back-pressure}
\label{sec-6}

\newcommand{\pmelange}{p_{\text{melange}}}
Let $\pmelange$ be the additional melange back-pressure. Then $\psw \le \psw +
\pmelange \le \pice$. Put another way,
\begin{equation}
\label{eq:13}
0 \le \pmelange \le \pice - \psw.
\end{equation}

Let $\lambda$ be the ``melange back-pressure fraction'' (or ``relative
melange density'') ranging from $0$ to $1$, so that
\begin{equation}
\label{eq:14}
\pmelange = \lambda \cdot (\pice - \psw).
\end{equation}

Then the modified pressure imbalance term is
\begin{equation}
\label{eq:15}
\int_{b}^{h}(\pice - (\psw + \lambda(\pice - \psw)))\, dz = (1 - \lambda) \int_{b}^{h} (\pice - \psw)\, dz
\end{equation}

\end{document}